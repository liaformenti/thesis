% ==================================================
% CHAPTER 5: Summary and outlook %
% ==================================================

\chapter{Summary and outlook}
\label{chap:summary}
% Edit count: Lia - 0, Brigitte - 0

The cosmic muon dataset was used to independently confirm the absolute local offsets measured by the x-ray method. The absolute local offsets are currently the main input to the final link in the sTGC alignment scheme of the NSWs: the NSW alignment system monitors the position of alignment platforms on the surface of sTGC wedges, and the x-ray measurements provide the offset of the strip pattern at positions on the alignment platforms. The continuation of this analysis is detailed next (section~\ref{sec:outlook}) before considering the larger context (section~\ref{sec:importance}). 

% --------------------------------------------------
\section{Outlook}
% --------------------------------------------------
\label{sec:outlook}

Next all quadruplets with suitable cosmics data should be surveyed (first) to flag anomalous quadruplets. If the correlation plot like figures~\ref{fig:correlation} or \ref{fig:no_correlation} reveals an unexpected correlation or has a large scatter, it would indicate an issue with either the cosmics or x-ray data collection to be investigated further. The uncertainty in each set of tracking points would inform the interpretation of the results. Also, the quality of the correlation should be evaluated over all quadruplets instead of individually. 
 
For now, the results for the individual quadruplets tested support the use of the x-ray data to build a global misalignment model~\cite{lefebvre_precision_2020}. Work on creating a misalignment model is ongoing with the development of \package{stgc\_as\_built\_fit}~\cite{lefebvre_stgc_as_built_fit}. Currently, the algorithm compares the y-position of a local group of strips at each x-ray gun position as measured by the x-ray and CMM methods in a $\chi^2$ fit. The CMM measurements were taken before the cathode boards were assembled into quadruplets, so alignment parameters for the given layer were extracted from the $\chi^2$ fit by stepping the CMM y-value towards the x-ray y-value by adjusting the alignment parameters (currently, a rotation and global offset). The plan is that these misalignment parameters will be provided to \package{Athena}~\cite{the_atlas_collaboration_athena} to correctly reconstruct precision muon tracks from the NSWs' sTGCs.

Figure~\ref{fig:correlation} and \ref{fig:no_correlation} clearly show that the uncertainty in the mean cosmics residuals was much smaller than the uncertainty in the x-ray residuals. It was the result of the uncertainty in the cosmic muon hit positions on each layer being smaller than the uncertainty on the x-ray beam profile centers (\SI{60}{\micro\meter} versus \SI{120}{\micro\meter}). Therefore, it would be great to use the cosmics residuals as input to calculate and reduce the uncertainty on the misalignment parameters. Since mean cosmics residuals can only provide relative misalignment information, one idea would be to use them to constrain the fit of the alignment parameters. In this case, the alignment parameters would need to be fitted on all layers at once, and the shifting y-values on each layer forced to create an abstracted track residual equal to the mean cosmics residuals (within uncertainty). Or, instead of constraining the fit, it could be penalized if the resulting parameters do not result in abstracted track residuals equal to the mean cosmics residuals within uncertainty. 

% --------------------------------------------------
\section{Importance in context}
% --------------------------------------------------
\label{sec:importance}

Ultimately, the position of the sTGC strip electrodes need to be known in ATLAS to within ~\SI{100}{\micro\meter} so that they can provide the required position resolution for precision muon tracking after the high-luminsity LHC upgrade. The x-ray measurments are the key dataset being used to achieve this goal given the misalignments between sTGC layers. The cosmics dataset was used to confirm the local offsets measured with the x-ray gun, and could be used to improve the misalignment model that will position each strip.

Achieving the required position resolution on each tracking layer of the NSWs in the particle track bending plane achieves the design momentum resolution of the NSWs for muons ejected towards the endcaps of ATLAS. These muons will be important signatures of electroweak and Higgs sector events of interest for the Collaboration's future physics goals~\cite{nsw_tdr}. Being the second of two tracking technologies on the NSWs, an effective misalignment model of sTGC quadruplets is a necessary part of making the NSWs redundant for 10 or more years of recording collisions in the High Luminosity era of the LHC. 