% ==================================================
% CHAPTER 6: Comparing cosmic muon and x-ray relative strip position offsets %
% ==================================================

%TODO : Add the x-ray residual arrow overlay on the cosmics residual TH2. Maybe restructure so all the comparison is in this chapter entirely. 
%TODO : Add this idea in this chapter: "Note that the mean of cosmics residuals around the x-ray points were calculated in bins exactly centered on the nominal x-ray gun position, unlike in figure~\ref{fig:res_mean_th2}."
\chapter{Comparing cosmic muon and x-ray relative strip position offsets}
\label{chap:comparison}
% Edit count: Lia - 1, Brigitte - 0

The goal was to validate the local offsets extracted from the x-ray data with cosmics data. The complication was that the x-ray dataset provided absolute local offsets while the cosmics dataset provided relative local offsets, which could not be compared directly. The solution was to use the x-ray local offsets to calculate relative local offsets. The x-ray relative local offset is the x-ray residual reconstructed from an abstract track using the beam profile centers on each layer as the track hits. The cosmics relative local offset was taken as the Gaussian mean of muon track residuals in a \SI{100}{mm} by \SI{100}{mm} area, referred to the as the mean cosmics residual. Relative local offsets of each type calculated using the same reference layers are compared for each area where x-ray data is available. The  results of the comparison are presented here.

% --------------------------------------------------
% \section{Evaluating}
% --------------------------------------------------


%---------------------------------------------------
\section{Assessing correlation}
%---------------------------------------------------
\label{sec:assessing_correlation}

The 2D visualizations of the mean cosmics and x-ray residuals for tracks on layer 2 with reference layers 1 and 3 on quadruplet-1 and quadruplet-2 are shown in figure~\ref{fig:res_compare_th2}. Figure~\ref{fig:res_compare_th2} is a superposition of figures~\ref{fig:res_mean_th2} and \ref{fig:xray_res_th2}.

\newpage
\thispagestyle{empty}
\newgeometry{top=0.5in,bottom=0.5in}
\begin{figure}
\centering
\begin{subfigure}{\textwidth}
  \centering
  \includegraphics[width=\linewidth]{figures/QL2P11_compare_residuals_th2_layer2_fixedlayers13.pdf}
  \caption{Quadruplet-1 residuals of tracks on layer 2, reference layers 1 and 3.}
  \label{fig:res_compare_th2_ql2p11}
\end{subfigure}%
\vspace*{\floatsep}
\begin{subfigure}{\textwidth}
  \centering
  \includegraphics[width=\linewidth]{figures/QL2P08_compare_residuals_th2_layer2_fixedlayers13.pdf}
  \caption{Quadruplet-2 residuals of tracks on layer 2, reference layers 1 and 3.}
  \label{fig:res_compare_th2_ql2p8}
\end{subfigure}
\caption{The mean cosmics residuals are shown using colour. The x-ray residuals available at nominal gun positions are drawn as arrows and the value of the residual annotated in millimeters with uncertainty $\pm$\SI{0.15}{mm}. The length of the arrows is 500 times the value of the x-ray residual, scaled for visibility. These plots are a superposition of figures~\ref{fig:res_mean_th2} and \ref{fig:xray_res_th2}.}
\label{fig:res_compare_th2}
\end{figure}
\newpage
\restoregeometry

Figure~\ref{fig:res_compare_th2_ql2p11} shows that for quadruplet-1 the x-ray residuals are of the same sign and order as the mean cosmics residuals, as can be seen by comparing the the annotated value of the x-ray residual to the mean cosmics residual represented by colour; quadruplet-1's mean cosmics and x-ray residuals are correlated to some degree. For quadruplet-2, the x-ray residuals are of the right order compared to the mean cosmics residuals, but the correlation is less apparent. While x-ray residuals do not reveal a pattern across the layer's surface, the mean cosmics residuals show a structure to the relative local offsets since they vary smoothly over the surface of layer 2. 

The comparison of mean cosmics and x-ray residuals was done for several quadruplets for all tracking combinations (not just layer 2 residuals calculated with fixed layers 1 and 3 like in figure~\ref{fig:res_compare_th2}). Scatter plots of the x-ray and mean cosmics residuals on quadruplet-1 and -2 for all tracking combinations shown in figures~\ref{fig:correlation} and \ref{fig:no_correlation} reveal the degree of correlation between the datasets. In the correlation plots, each rectangle is centered on the value of a mean cosmics and x-ray residual pair calculated with a given tracking combination for every gun position where data is available; the height and width of the squares are the uncertainty in the mean cosmics and x-ray residuals respectively. Note that in the scatter plots, the regions of interest where cosmics tracks are included in the calculation of mean of residuals are exactly centered on the nominal x-ray beam position, unlike in figure~\ref{fig:res_compare_th2}.

% YOU ARE HERE

\begin{figure}
    \centering
    \includegraphics[width = \textwidth]{figures/figure_QL2P11_3100V_2021-08-05_QL2P11_local_cosmic_and_xray_data_correlation_plot.pdf}
    \caption{Correlation plot between x-ray and mean cosmics residuals for all tracking combinations for quadruplet-1. Each rectangle is centered on an x-ray and mean cosmics residual pair calculated at a given gun position and for a certain tracking combination. The width of the rectangles in $x$ and $y$ are the uncertainty in the x-ray and mean cosmics residual respectively. A printer-friendly version of this plot is available in appendix~\ref{appendix:print}.}
    \label{fig:correlation}
\end{figure}

The fitted slope and offset in figure~\ref{fig:correlation} show that the two quadruplet-1 datasets are correlated. The large uncertainty on the x-ray residuals set a limit on the sensitivity of the analysis, for if the absolute value of the x-ray residuals of a quadruplet were smaller than the x-ray residual uncertainties, no conclusion about the correlation could be drawn, like for quadruplet-2 (figure~\ref{fig:no_correlation}). This result is reflected in the small x-ray residuals shown in figure~\ref{fig:res_compare_th2_ql2p8} that do not reveal a pattern in the relative local offsets across the surface of layer 2. However, figure~\ref{fig:no_correlation} shows that the x-ray and mean cosmics residuals are centered around zero, as is expected for a quadruplet with small relative misalignments between layers.

\begin{figure}
    \centering
    \includegraphics[width = \textwidth]{figures/figure_QL2P08_3100V_2021-08-16_QL2P08_local_cosmic_and_xray_data_correlation_plot.pdf}
    \caption{Correlation plot between x-ray and mean cosmics residuals for all tracking combinations for quadruplet 2. Each rectangle is centered on an x-ray and mean cosmics residual pair calculated at a given gun position and for a certain tracking combination. The width of the rectangles in $x$ and $y$ are the uncertainty in the x-ray and mean cosmics residual respectively. A printer-friendly version of this plot is available in appendix~\ref{appendix:print}.}
    \label{fig:no_correlation}
\end{figure}

There are three patterns in the residuals on the scatter plot explained by geometry. First, for both datasets the uncertainty in the extrapolated track residuals were larger than the interpolated track residuals because of the extrapolation lever arm. For the x-ray residuals, the effect of the lever arm on the uncertainty was direct since the residual was calculated from a single abstract track; for the mean cosmics residuals it was the widening of the residual distribution due to the extrapolation lever arm that increased the uncertainty in the fitted mean of residuals. Second, residuals calculated through extrapolation tend to be larger because the extrapolation lever arm can produce more extreme values of the track position on the layer of interest. Third, the points in figure~\ref{fig:correlation} are geometrically correlated (e.g. they seem to be roughly mirrored around the origin). This is expected since the residuals calculated using a given set of three layers should be geometrically correlated by the local offsets on the fixed layers and the layer of interest (the $d_{local, i}$ on each layer as defined in equation~\ref{eqn:local_translation}). 

% --------------------------------------------------
\section{Discussion}
% --------------------------------------------------

Several quadruplets were tested for each quadruplet construction geometry built in Canada. Each quadruplet fell into one of the two categories: residuals large enough to see a correlation, or residuals too small to see a correlation. Since the x-ray and mean cosmics residuals were measures of the relative local offsets between the layer and the two reference layers, quadruplets with the largest relative misalignments had the largest range of residuals. the correlation plots were an easy visual way to identify quadruplets with large relative misalignments.

The most significant limit on measuring the degree of correlation between the x-ray and mean cosmics residuals was the uncertainty on the x-ray residuals, which stemmed from the systematic uncertainty of \SI{120}{\micro\meter} in the x-ray beam profile centers used to build the abtract tracks. For example, in figure~\ref{fig:no_correlation} the uncertainty in the x-ray residuals makes detecting correlation impossible. The x-ray method was limited primarily by the systematic uncertainties in the relative alignment of the platforms and the gun, especially the gun angle.

The analysis of certain quadruplets was limited by the availability of data. Sometimes, less than three layers were surveyed for a given x-ray gun position so no residuals could be calculated. Too few x-ray residuals prevented the analysis from detecting a significant correlation, should it even be measurable. Often, the analysis of smaller quadruplets (placed innermost on the wheel) suffered as a result because they had fewer alignment platforms, and hence gun positions, on their surfaces. The analysis was also limited to certain quadruplets. The wedges constructed the earliest (typically small wedges) were surveyed when the x-ray method was still being designed and so have limited x-ray residuals calculated from beam profiles of lower quality. In addition, not all cosmic muon test sites had enough front end electronics to collect data on three layers simultaneously, which is the minimum required to be able to calculate residuals.

Nonetheless, the comparison of x-ray and cosmics residuals was really to confirm the x-ray method's ability to measure local offsets with an independent dataset. The x-ray local offsets allow the calculation of relative local offsets that have been correlated to the cosmics relative local offsets. Therefore, the analysis of quadruplets with relative local offsets large enough to detect a correlation validates the x-ray method's ability to measure local offsets. 

The potential of using relative local offsets calculated from cosmics data to study relative alignment between sTGC layers stands on its own. For example, although the x-ray relative local offsets of quadruplet-2 in figure~\ref{fig:res_compare_th2_ql2p8} do not reveal a pattern, the variation in the cosmics relative local offsets do. Identifying the pattern is possible because mean cosmics residuals can be calculated across the entire area and are sensitive to smaller relative local offsets since their uncertainty is significantly smaller. 

The advantage of the x-ray dataset over the cosmics dataset is that absolute local offsets are measurable thanks to the reference frame provided by the alignment platforms. This is required to measure the position of strips in the ATLAS coordinate system to satisfy the NSWs' precision tracking goals. The x-ray local offsets are being used to build an alignment model of strips in each quadruplet. It is compelling to imagine using the cosmics relative local offsets to improve the model considering their precision and ability to capture effects across the entire area of the quadruplet.