% ==================================================
% CHAPTER 5: Using x-rays to measure relative strip position offsets
% ==================================================

\chapter{Using x-rays to measure relative strip position offsets}
\label{chap:xray}

%TODO : consistently call the x-ray centroids the "x-ray beam profile centers"

Other work on characterizing relative alignments between quadruplet layers has been completed~\cite{zhao_cosmic_2019} or is ongoing, \textcolor{red}{(Can I cite John's thesis-in-progress?)} but what is required are the absolute strip positions with respect to their nominal position in the ATLAS analysis coordinate system to be input into \package{Athena}~\cite{the_atlas_collaboration_athena}. Somehow, alignment parameters must be derived to create a model of absolute strip positions - which is not possible with the cosmics dataset. Absolute local offset measurements were done by the so-called x-ray method. 

The author did not partake in the design of or data collection with this method. The reader is referred to the paper describing the x-ray method~\cite{lefebvre_precision_2020}, although the experimental setup has changed slightly since it was written. The experimental setup described here is the most current and was used to collect the data used in this thesis.

% NOTE: Wrote this with information available in JINST, or things I know to be updated (eg. brass holder and collimator). I can guess at other things based on the WedgeAlignment-Production code, but don't actually know. Those things are in iffalse statements.
% --------------------------------------------------
\section{Experimental setup}
% --------------------------------------------------

The x-ray tests were performed after the quadruplets arrived at CERN, were assembled into wedges, and alignment platforms installed. Essentially, an x-ray gun was attached to one of the alignment platforms glued to the surface of the wedge and the beam profile recorded by the strips.

\begin{figure}
    \centering
    \includegraphics[width = 0.5\textwidth]{figures/figure_xray_setup.pdf}
    \caption{The x-ray gun mounted to the alignment platform on the surface of the wedge. Adapted from~\cite{lefebvre_precision_2020}.}
    \label{fig:xray_setup}
\end{figure}

The wedges were installed on carts that could rotate their surface to a horizontal position. A mounting platform was installed on top of the alignment platform using a three-ball mount. The x-ray gun used was an \href{https://www.amptek.com/-/media/ametekamptek/documents/resources/specs/mini-x-specifications.pdf?la=en\&revision=512f7eb3-01b3-47fd-864f-5525c850fc6e\&hash=B8B03C0592486E2D91C566C4326F15F5}{Amptek Mini-X tube}. The gun was placed in a brass holder with built-in \SI{2}{mm} collimator and \SI{280}{\micro\meter} copper filter that be mounted on one of five positions on the mounting platform, as shown in figure~\ref{fig:xray_setup}. Gun positions were chosen to avoid wire support structures in the sTGCs that reduce hit efficiency~\cite{lefebvre_thesis} and boundaries between sets of strips readout by two different ASICs (each ASIC had 64 channels that could be connected to electrodes) that could each have different thresholds. 

As with cosmics data collection, each sTGC also needed gas and high voltage to operate. Each layer was operated at \SI{2.925}{kV} with high voltage from a NIM crate. The chambers were flushed with CO$_2$ before and during data collection.

% The copper filter helped to reduce the effect of attenuation non-uniformities.
The gun produced x-rays with energies under \SI{40}{\kilo\electronvolt} with peaks in the 7-\SI{15}{keV} range. \iffalse Peaks in the 0-\SI{30}{keV} range were filtered out by the copper filter and the copper of the sTGCs. \fi The x-rays mostly interacted with the wedge's copper electrodes and gold-plated tungsten wires via the photo effect. The resulting photoelectrons caused ionization avalanches that are picked up by the strips.
% During ATLAS operation, the position of the source plates will be monitored using the new alignment system~\cite{nsw_tdr}. Therefore, their position will be known in the absolute ATLAS coordinate system. 

% --------------------------------------------------
\section{Data acquisition}
% --------------------------------------------------

A different version of the same front end electronics, but the same ASIC, as used in cosmics testing were used for the x-ray testing to amplify the data and measure the peak signal amplitude. Data was collected for two minutes per gun position with random triggers. A trigger recorded all signals above threshold \iffalse within \SI{75}{ns} and the signals on neighbouring strips. \fi Pad and wire data was not recorded.
% Neighbour triggering based on my understanding of the analysis. Does not actually matter for these purposes.

% --------------------------------------------------
\section{Data preparation}
% --------------------------------------------------

Like with cosmics analysis, a default pedestal was subtracted from the signal peak amplitude of each hit.

Clusters were defined as groups of contiguous strip hits collected within \SI{75}{ns}. The peak signal amplitude of each electrode in a cluster was fit with a Gaussian, and the mean of the Gaussian was taken as the cluster position. Custer positions are corrected for DNL (see definition in appendix~\ref{appendix:systematics_dnl}). Only clusters composed of hits on 3-5 strips were used in the x-ray analysis. Clusters with signal on more than 5 strips were cut because they were most likely caused by photoelectrons ejected with enough energy to  \iffalse cause more primary ionization and subsequent avalanches ( \fi be $\delta$-rays.
% Note: analysis definitely does the double cluster cut

The x-ray analysis must diverge entirely from the cosmics analysis because the x-rays do not leave tracks. The signals picked up by the strips are from photoelectrons generated on the metals of the sTGCs that only travel through one gas volume and are ejected at all angles. Instead of creating tracks, the cluster position distribution is used to define the beam profile. A typical beam profile is shown in figure~\ref{fig:xray_beam_profile}.

\begin{figure}
    \centering
    \includegraphics[width = 0.5\textwidth]{figures/figure_xray_beam_profile.pdf}
    \caption{Distribution of x-ray cluster mean positions after the analysis cuts and corrections. The strip cluster multiplicity, $m$, was limited to 3, 4 and 5. The red line is a Gaussian fit of the distribution and the pink lines denote the edges of the strips~\cite{lefebvre_precision_2020}.}
    \label{fig:xray_beam_profile}
\end{figure}

% --------------------------------------------------
\section{Measuring local offsets}
% --------------------------------------------------
The mean of the cluster position distribution was taken as the x-ray beam profile center. The expected center was calculated assuming a wedge with nominal geometry given the gun position, corrected for: the geometry of the brass holder, the positioning and angle of the source plates and the beam angle. The difference between the expected and reconstructed beam profile center is a measure of the local offset. Applying the logic of equation~\ref{eqn:local_translation} to the beam profile, the fitted mean acts as $y$, the expected center is $y_{nom}$ and the local offset is $d_{local}$ as before. The x-ray local offsets  give the absolute local position of the strip pattern with respect to the source plates. Since the position of the source plates will be monitored by the alignment system in ATLAS~\cite{nsw_tdr}, the local position of the strip pattern can be known in the ATLAS coordinate system for every position where x-ray data was taken.

%TODO : Maybe move this paragraph to start of comparison chapter
The main advantage of the x-ray dataset over the cosmics dataset is that absolute local offsets are measurable thanks to the reference frame provided by the source plates. However, the systematic uncertainty on the x-ray offsets is large: \SI{120}{\micro\meter} was accepted by the collaboration. In addition, local offset measurements were limited to the positions of the alignment platforms; only 10 - 20 positions were surveyed for each wedge. Therefore, validating the x-ray measurements and seeing how they can be improved is important because of the uncertainty in and incompleteness of the dataset. Since the local offsets for x-rays and cosmics data are measured in different coordinate systems, they cannot be compared directly. Bringing the cosmics relative local offsets into an absolute coordinate system is impossible; however, the x-ray local offsets can be brought into a relative coordinate system.

% --------------------------------------------------
\section{Measuring relative local offsets}
% --------------------------------------------------
The measured x-ray beam profile centers were systematically affected by local offsets in the same way as the mean cosmics residuals, as modeled by equation \ref{eqn:local_translation}. Therefore, if a 2-layer track is abstracted from the beam profile centers on each layer and the residual calculated on a third layer, that residual should match the local mean cosmics residual. 

The track is "abstracted" because a so-called "beam profile center" is actually the Gaussian mean of all selected mean cluster positions recorded during the x-ray data taking period. Abstracting a track was necessary because the x-rays cause signal in the chamber via the photoeffect so there were not individual "x-ray tracks" to record. In fact the x-ray data could be collected separately for each layer. Nonetheless, since the effect of local offsets on the beam profile centers was the same as their effect on the recorded cosmics cluster positions the difference in algorithm between x-ray and cosmics analysis was allowed. 

For each x-ray survey position, the x-ray residual was calculated for all possible tracking combinations (which required an x-ray beam profile on at least three layers). The position of the x-ray residuals are shown as black dots over figure~\ref{fig:res_mean_th2_L2_F13} and \ref{fig:res_mean_th2_L4_F13}. Note that the mean of cosmics residuals around the x-ray points were calculated in bins exactly centered on the nominal x-ray gun position, unlike in figure~\ref{fig:res_mean_th2}.

The uncertainty on the x-ray residuals was the error propagated through the tracking, taking an uncertainty of \SI{120}{\micro\meter} on each beam profile center. The uncertainty on the x-ray residuals ranged from \SI{0.1}{mm} to \SI{0.4}{mm} for the most to least geometrically favourable tracking combination - significantly larger than the uncertainty on the mean cosmics residuals. 
%TODO : transition to next chapter
%TODO : Should I present the x-ray residuals somehow? Eg. repeat figure res_mean_th2 but with x-ray residuals written on top? <-- This is my favourite option so far