% ==================================================
% CHAPTER 1: Introduction %
% ==================================================

\chapter{Introduction}
\label{chap:intro}
% Edit count: Lia - 0, Brigitte - 0

% Miscellaneous intros
%The primary goal in building the Large Hadron Collider (LHC) and the ATLAS experiment was to search for a Higgs boson
%Studying particle detectors is interesting because of the interplay between the physics of what is to be studied with the detector and the physics of how the detector works. 
%Particle detectors intertwine physics concepts at multiple scales since their use requires understanding both the physics of how they work and the physics of what they are meant to study. 
%The details of how a particle detector works intertwine with the physics it is meant to study. Especially in a collaboration as large as ATLAS. 
%Small-strip thin gap chambers (sTGCs) for the ATLAS experiment at CERN 
%The questions proposed in the first recorded physics case for the Large Hadron Collider (LHC) are still only partially answered~\cite{brianti_large_1984}. it is clear there is still more to study at the LHC if the study of the %standard model is to continue. 
%The High-Lumnosity Large Hadron Collider (HL-LHC) project was approved to combat the plateau in statistial gain of recording particle collisions at the LHC at CERN~\cite{hl_lhc_tdr}. 
%The questions proposed in the first recorded physics case for the Large Hadron Collider (LHC) are still only 
%If the study of particle physics is to continue, studying the Higgs boson 

% making it an indispensable tool for studying particle physics interactions in an environment as close as can be simulated to the early universe.

The Standard Model (SM) is a theoretical framework that describes experimental observations of particles and their interactions at the smallest distance scales; however, the questions the SM does not address motivate more experimentation. 

Accelerators collide particles to generate interactions that can be recorded by detectors for further study. Detectors measure the trajectory and energy of all secondary particles produced in collisions to understand the interaction. The Large Hadron Collider (LHC)~\cite{evans_lhc_2008} at CERN is the world's most energetic particle accelerator. Its energy makes it a unique tool to study elementary particles and their interactions in an environment with conditions similar to what would have existed in the early universe. If study at the energy frontier is to continue, the LHC must go on.

After 2025, the statistical gain in running the LHC further without significant increase in beam intensity will become marginal. The High Luminosity Large Hadron Collider (HL-LHC) project~\cite{hl_lhc_tdr} is a series of upgrades to LHC infrastructure that will allow the LHC to collect approximately ten times more data than in the initial design by $\sim$2030. The increase in LHC beam intensity will result in a large increase in collision rate that will make accessible and improve statistics on several measurements of interest~\cite{dainese_physics_2018}, many only possible at the LHC and the energy frontier. The increase in beam intensity will also increase the level of background radiation, requiring major upgrades to the experiments used to record the outcomes of the particle collisions.

The ATLAS experiment~\cite{collaboration_atlas_2008} is one of the LHC's general-purpose particle detector arrays, positioned around one of the collision points of the LHC.  During the 2019-2022 Long Shutdown of the LHC, the most complex upgrade of the ATLAS experiment is the replacement of the small wheels of the muon spectrometer with the so-called New Small Wheels (NSWs)~\cite{nsw_tdr}. The detector upgrade addresses both the expected decrease in hit efficiency of the precision tracking detectors and the high fake trigger rate expected in the muon spectrometer at the HL-LHC. The NSWs are made of two different detector technologies: micromegas and small-strip thin gap chambers (sTGCs). Micromegas are optimized for precision tracking while sTGCs are optimized for rapid triggering, although each will provide complete coverage and measurement redundancy over the area of the NSWs. Eight layers each of sTGCs cover the NSWs. Practically, countries involved in detector constructor created quadruplet modules of four sTGCs glued together that were arranged and installed over the area of the NSWs once they arrived at CERN. Teams across three Canadian institutions built and characterized 1/4 of all the required sTGCs.

The sTGCs are gas ionization chambers that consist of a thin volume of gas held between two cathode boards. One board is segmented into strip readout electrodes of 3.2 mm pitch.  The position of the particle track in the precision coordinate can be reconstructed from the strip signals~\cite{nsw_tdr}. The sTGCs acheived the design track spatial resolution in the precision coordinate of less than \SI{100}{\micro\meter} per detector plane that will allow them to achieve a 1 mrad track angular resolution using the 8 layers of sTGC on the NSW~\cite{abusleme_performance_2016, nsw_tdr}. The NSW measurement of the muon track angle will be provided to the  ATLAS trigger and used to reject tracks that do not originate from the interaction point~\cite{nsw_tdr}.

The precise measurement of  a muon track angle depends on knowing the position of each readout strip within the  ATLAS coordinate system.  To achieve this, the position of specific locations on the surface of sTGC quadruplets will be monitored by the ATLAS alignment system to account for time-dependent deformations~\cite{nsw_tdr}. Within a quadruplet module, the strip positions could have been shifted off of nominal by non-conformities of the strip pattern etched onto each cathode boards~\cite{carlson_results_2019} and shifts between strip layers while gluing sTGCs into quadruplets.

An xray gun was used to measure the offset of strips from their nominal position at the locations that will be monitored by the ATLAS alignment system thereby providing, locally, an absolute “as-built” strip position within the ATLAS coordinate system.  Estimates of the “as-built” positions of every readout strip are obtained by building an alignment model from the available x-ray measurements~\cite{lefebvre_precision_2020}.

The technique of measuring the “as-built” strip positions using xray data has never been used before and must be validated. This thesis describes the use of cosmic muon data, recorded to characterize the performance of each Canadian-made sTGC module, to validate the x-ray strip position measurements.  A description of how this work fits within the overall alignment scheme of the NSW is also presented. 

Chapter~\ref{chap:hep} gives a brief overview of high energy particle physics necessary to understand the physics motivation of the HL-LHC and NSW upgrades. Chapters~\ref{chap:lhc_atlas} and~\ref{chap:nsw} present additional details on the LHC, ATLAS, the NSWs, and sTGCs. In Chapter~\ref{chap:cosmics}, the cosmic ray testing procedure and how the position of the strips can be probed with cosmics data is presented. Chapter~\ref{chap:xray} introduces the x-ray method, and in Chapter~\ref{chap:comparison}, the x-ray offsets are validated with cosmic muon data. The thesis concludes with a summary and outlook in Chapter~\ref{chap:outlook_and_summary}.

% In this chapter, the experimental setup of the ATLAS experiment at the LHC is presented, and the NSW upgrade is motivated and detailed before later chapters explain how sTGC characterization datasets are being used for alignment, with a focus on the cosmic muon dataset.
