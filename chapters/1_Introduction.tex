% ==================================================
% CHAPTER 1: Introduction %
% ==================================================

\chapter{Introduction}
\label{chap:intro}
% Edit count: Lia - 0, Brigitte - 0

% Miscellaneous intros
%The Large Hadron Collider (LHC) and the ATLAS experiment were designed to search for a Higgs boson~\cite{atlas_letter_of_intent_1992}
%The primary goal in building the Large Hadron Collider (LHC) and the ATLAS experiment was to search for a Higgs boson
%Studying particle detectors is interesting because of the interplay between the physics of what is to be studied with the detector and the physics of how the detector works. 
%Particle detectors intertwine physics concepts at multiple scales since their use requires understanding both the physics of how they work and the physics of what they are meant to study. 
%The details of how a particle detector works intertwine with the physics it is meant to study. Especially in a collaboration as large as ATLAS. 
%Small-strip thin gap chambers (sTGCs) for the ATLAS experiment at CERN 
%The questions proposed in the first recorded physics case for the Large Hadron Collider (LHC) are still only partially answered~\cite{brianti_large_1984}. it is clear there is still more to study at the LHC if the study of the %standard model is to continue. 
%The High-Lumnosity Large Hadron Collider (HL-LHC) project was approved to combat the plateau in statistial gain of recording particle collisions at the LHC at CERN~\cite{hl_lhc_tdr}. 
%The questions proposed in the first recorded physics case for the Large Hadron Collider (LHC) are still only 
%If the study of particle physics is to continue, studying the Higgs boson 

The High-Lumnosity Large Hadron Collider (HL-LHC) project was approved to combat the plateau in statistial gain of recording particle collisions at the LHC at CERN~\cite{hl_lhc_tdr}. Being the most energetic particle accelerator, the LHC still offers unique physics opportunities for studying the Higgs and electroweak sectors of the standard model\cite{dainese_physics_2018}; if the study at the energy frontier is to continue, the LHC must go on. The HL-LHC upgrade aims to increase the luminosity of the LHC by up to a factor of 7 in the next 10 years, which ultimately increases the number of meaningful collisions. Naturally, various sub-systems of the experiments used to capture the outcomes of the collisions will require upgrades to handle higher collision rates and background radiation rates than they were designed for. 

The ATLAS experiment~\cite{collaboration_atlas_2008} is one of the LHC's general-purpose particle detector arrays. The largest upgrade the ATLAS experiment will undergo is the replacement of the small wheels of the muon spectrometer with the so-called New Small Wheels (NSWs)~\cite{nsw_tdr}. The NSW addresses both the decrease in hit efficiency in the precision tracking detectors of the current small wheel expected because of the increased hit rate and the high fake trigger rate in the endcaps of the muon spectrometer. Two different detector technologies will be installed, stacked on the NSW frame: micromegas (MM), and small-strip thin gap chambers (or sTGCs). MM are micro-mesh gaseous

In this work, the main dataset used to correct for the internal alignment of sTGC components (the x-ray dataset~\cite{lefebvre_precision_2020}) is validated with characterization data collected on the sTGCs using cosmic muons. How the internal alignment fits into the overall alignment system is also detailed. 

In this chapter, the experimental setup of the ATLAS experiment at the LHC is presented, and the NSW upgrade is motivated and detailed before later chapters explain how sTGC characterization datasets are being used for alignment, with a focus on the cosmic muon dataset.
%TODO : Beef this up