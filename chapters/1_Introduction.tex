% ==================================================
% CHAPTER 1: Introduction %
% ==================================================

\chapter{Introduction}
\label{chap:intro}
% Edit count: Lia - 0, Brigitte - 0

% Miscellaneous intros
%The Large Hadron Collider (LHC) and the ATLAS experiment were designed to search for a Higgs boson~\cite{atlas_letter_of_intent_1992}
%The primary goal in building the Large Hadron Collider (LHC) and the ATLAS experiment was to search for a Higgs boson
%Studying particle detectors is interesting because of the interplay between the physics of what is to be studied with the detector and the physics of how the detector works. 
%Particle detectors intertwine physics concepts at multiple scales since their use requires understanding both the physics of how they work and the physics of what they are meant to study. 
%The details of how a particle detector works intertwine with the physics it is meant to study. Especially in a collaboration as large as ATLAS. 
%Small-strip thin gap chambers (sTGCs) for the ATLAS experiment at CERN 
%The questions proposed in the first recorded physics case for the Large Hadron Collider (LHC) are still only partially answered~\cite{brianti_large_1984}. it is clear there is still more to study at the LHC if the study of the %standard model is to continue. 
%The High-Lumnosity Large Hadron Collider (HL-LHC) project was approved to combat the plateau in statistial gain of recording particle collisions at the LHC at CERN~\cite{apollinari_high-luminosity_2017}. 
%The questions proposed in the first recorded physics case for the Large Hadron Collider (LHC) are still only 
%If the study of particle physics is to continue, studying the Higgs boson 

\section{The Large Hadron Collider and the ATLAS experiment}
\textcolor{red}{Can I copy-paste the usual ATLAS detector section?}

\subsection{The ATLAS muon spectrometer}
The outermost detector subsystem of ATLAS is the muon spectrometer since muons, being minimum ionizing particles, are not stopped by the calorimeters.