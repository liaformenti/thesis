% ==================================================
% CHAPTER 1: Introduction %
% ==================================================

\chapter{Introduction}
\label{chap:intro}
% Edit count: Lia - 0, Brigitte - 0

% Miscellaneous intros
%The Large Hadron Collider (LHC) and the ATLAS experiment were designed to search for a Higgs boson~\cite{atlas_letter_of_intent_1992}
%The primary goal in building the Large Hadron Collider (LHC) and the ATLAS experiment was to search for a Higgs boson
%Studying particle detectors is interesting because of the interplay between the physics of what is to be studied with the detector and the physics of how the detector works. 
%Particle detectors intertwine physics concepts at multiple scales since their use requires understanding both the physics of how they work and the physics of what they are meant to study. 
%The details of how a particle detector works intertwine with the physics it is meant to study. Especially in a collaboration as large as ATLAS. 
%Small-strip thin gap chambers (sTGCs) for the ATLAS experiment at CERN 
%The questions proposed in the first recorded physics case for the Large Hadron Collider (LHC) are still only partially answered~\cite{brianti_large_1984}. it is clear there is still more to study at the LHC if the study of the %standard model is to continue. 
%The High-Lumnosity Large Hadron Collider (HL-LHC) project was approved to combat the plateau in statistial gain of recording particle collisions at the LHC at CERN~\cite{hl_lhc_tdr}. 
%The questions proposed in the first recorded physics case for the Large Hadron Collider (LHC) are still only 
%If the study of particle physics is to continue, studying the Higgs boson 

%TODO : Start with the physics motivation of particle physics and accelerators, then LHC (energy frontier), then HL-LHC (BV, 2021-09-23)
The High-Lumnosity Large Hadron Collider (HL-LHC) project~\cite{hl_lhc_tdr} was approved to combat the plateau in statistial gain of recording particle collisions at the LHC~\cite{evans_lhc_2008} at CERN. Being the most energetic particle accelerator, the LHC still offers unique physics opportunities for studying the Higgs and electroweak sectors of the standard model\cite{dainese_physics_2018}; if the study at the energy frontier is to continue, the LHC must go on. The HL-LHC upgrade aims to increase the luminosity of the LHC by up to a factor of 7 in the next 10 years, which ultimately increases the number of meaningful collisions. Naturally, various sub-systems of the experiments used to capture the outcomes of the collisions will require upgrades to handle higher collision rates and background radiation rates than they were designed for. 

The ATLAS experiment~\cite{collaboration_atlas_2008} is one of the LHC's general-purpose particle detector arrays, positioned around one of the collision points of the LHC. It detects the products of LHC collisions. During the 2019-2022 Long Shutdown of the LHC, the most complex upgrade of the ATLAS experiment is the replacement of the small wheels of the muon spectrometer with the so-called New Small Wheels (NSWs)~\cite{nsw_tdr}. The NSW upgrade addresses both the expected decrease in hit efficiency in the precision tracking detectors of the current small wheel expected and the high fake trigger rate of the muon spectrometer. Two different detector technologies will be installed, stacked on the NSW frame: micromegas (MMs) and small-strip thin gap chambers (sTGCs). MMs are optimized for precision tracking while sTGCs are optimized for rapid triggering, although each will provide complete coverage and redundancy over the area of the NSW. Canada was responsible for providing 1/4 of the required sTGCs.

To reduce the fake trigger rate, the NSW will provide better track angular resolution to the ATLAS trigger system to reject tracks that do not originate from the collision~\cite{nsw_tdr}. sTGCs provide \SI{100}{\micro\meter} position resolution per detector plane~\cite{abusleme_performance_2016}, and are stacked in four (called an sTGC quadruplet) to provide \SI{1}{mrad} angular resolution on tracks~\cite{nsw_tdr, perez-codina_small-strip_2016}. To be fast enough to provide this information to the trigger, they were designed with the smallest number of readout electrodes~\cite{nsw_tdr}. sTGCs are gas ionization chambers where a thin volume of gas is held between two cathode boards. One boards is segmented into pad electrodes (of varying areas around \SI{300}{\centi\meter\squared}) and one is segmented into strip electrodes of \SI{3.2}{mm} pitch. The signals picked up by the pads due to a passing charged particle are used to select which strip electrodes to readout. The position of the particle track in the precision coordinate can then be reconstructed from the strip signals to provide the track angle to the trigger fast enough for the trigger to decide if the particle came from the interaction or not~\cite{nsw_tdr}.

% Both needed to provide position resolution better than \SI{100}{\micro\meter} per detector plane all incident angles and fast, efficient response in a high rate environment. MM are micro-mesh gaseous structures with \SI{100}{ns} response times and excellent position resolution for precision tracking. sTGCs are gas ionization chambers optimized to provide a fast response and, when stacked, an angular resolution better than \SI{1}{mrad} with the smallest number of readout electrodes for rapid triggering. Together, they provide a completely redundant system 

Precise position resolution is naught without accurate positioning of readout electrodes in ATLAS. The ATLAS alignment system is able to position the surface of three sTGC or MM quadruplets traversable by a muon track with respect to one another within \SI{40}{\micro\meter}. The internal geometry of the detectors must be controlled or corrected for to within the chambers' position resolution~\cite{nsw_tdr}. Corrections to the position of strip electrodes in sTGC quadruplets are in their final stages. The corrections are done with characterization data collected throughout the construction process. At the cathode board level, strip electrode positions are digitized with a coordinate measuring machine (CMM)~\cite{carlson_results_2019}. At the quadruplet level, sTGC quadruplets are characterized with cosmic rays and with an x-ray gun at positions that will be tracked by the alignment system. Cosmic muon data (cosmics data) can be used to measure relative strip position offsets in a chosen area with respect to the strip patterns on other layers, which is useful for quadruplet characterization but does not allow the strips to be positioned in the absolute ATLAS alignment system. The x-ray method~\cite{lefebvre_precision_2020} is able to measure offsets of the strip pattern near the x-ray gun in a coordinate system accessible to the alignment system; however, it is limited to a handful of positions on the surface of the quadruplet and should be validated by an independent method. In this work, cosmics data is used to measure relative strip offsets, the x-ray method is validated with cosmics data, and how this work fits into the overall alignment scheme is presented.

Chapters ~\ref{chap:lhc_atlas} and \ref{chap:nsw} give more details on ATLAS, the LHC, the NSW, and sTGCs required to understand the context of this work. In chapter~\ref{chap:cosmics}, the cosmic ray testing procedure and how the position of the strips can be probed with cosmics data is presented. Chapter~\ref{chap:xray} introduces the x-ray method, and in chapter~\ref{chap:comparison}, the x-ray offsets are validated with cosmic muon data. The thesis concludes with a summary and outlook in chapter~\ref{chap:outlook}.

% In this chapter, the experimental setup of the ATLAS experiment at the LHC is presented, and the NSW upgrade is motivated and detailed before later chapters explain how sTGC characterization datasets are being used for alignment, with a focus on the cosmic muon dataset.