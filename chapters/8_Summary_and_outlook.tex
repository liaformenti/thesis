% ==================================================
% CHAPTER 8: Summary and Outlook %
% ==================================================

\chapter{Summary and outlook}
\label{chap:outlook_and_summary}
% Edit count: Lia - 1, Brigitte - 0

The LHC~\cite{evans_lhc_2008} will be at the energy frontier of particle physics for at least the next decade, making it a unique tool with which to study particle physics. With the HL-LHC~\cite{hl_lhc_tdr}, high statistics on rare particle physics processes will enable more precise measurements of parameters of the Standard Model and increase the sensitivity to signatures of physics beyond the Standard Model~\cite{dainese_physics_2018}. To capitalize on the increased luminosity, the muon small wheels of the ATLAS experiment must be replaced to keep the current triggering and tracking performance~\cite{nsw_tdr}. 

sTGCs are gas ionization chambers optimized for a high rate environment~\cite{nsw_tdr}. Using the pad electrodes to define a region of interest makes it possible to get track segments of $\sim$\SI{1}{mrad} angular resolution quickly, which will be used as input to check if a collision originated from the interaction point and whether it should be triggered on~\cite{nsw_tdr, perez-codina_small-strip_2016}. Thanks to the careful design of the sTGCs, particularly the small pitch of the strip readout electrodes, the sTGCs are able to provide better than \SI{100}{\micro\meter} position resolution per detector plane to fulfill precision offline tracking requirements~\cite{abusleme_performance_2016}. 

Ultimately, the positions of the sTGC strip electrodes need to be known in ATLAS to within $\sim$\SI{100}{\micro\meter} so that they can deliver the required position resolution~\cite{nsw_tdr}. The strategy is to build an alignment model to estimate the position of each strip~\cite{lefebvre_precision_2020}. Input to the alignment model comes from the datasets used to characterize the quadruplets. The x-ray data~\cite{lefebvre_precision_2020} is the only characterization dataset that directly links the position of the strips to the ATLAS coordinate system. The alignment model could be built on x-ray data alone, but the sparseness of and large uncertainty on the local offsets mean that the alignment model could benefit from more input. The x-ray method is also a new technique that should be independently validated.

Relative local offsets measured with the cosmics and x-ray datasets were compared and the observed correlation confirmed the local offsets measured with the x-ray gun. Moreover, the cosmics relative local offsets are useful on their own. The 2D visualizations of relative local offsets make it possible to quickly identify areas of misaligned strips and make hypotheses as to the physical origin of those misalignments. Also, the cosmics residual means were shown to be robust and have uncertainties under \SI{100}{\micro\meter} for all two-fixed-layer reference frames, which is small in this context. Therefore, the cosmics relative local offsets complement the x-ray data by providing a complete, robust picture of the relative strip position offsets between layers. The next goal will be to use the cosmics relative local offsets to improve the alignment model and better position the sTGC strips in ATLAS.

Muons are important signatures of electroweak and Higgs sector events that physicists anticipate studying with a high-statistics dataset~\cite{dainese_physics_2018, nsw_tdr}. An effective alignment model of sTGC strip positions will ensure that the NSWs can be used to accomplish the ATLAS collaboration's physics goals during the High Luminosity era of the LHC.

% --------------------------------------------------
% \section{Importance in context}
% --------------------------------------------------
% \label{sec:importance}

%Ultimately, the positions of the sTGC strip electrodes need to be known in ATLAS to within $\sim$\SI{100}{\micro\meter} so that they can provide the required position resolution for the High-Luminosity LHC. The x-ray measurements will account for offsets of strips from their nominal position to achieve this goal. The cosmics dataset was used to confirm the local offsets measured with the x-ray gun, and could be used to improve the alignment model that will position each strip.

%Achieving the required position resolution on each layer of the NSWs in the particle track bending plane achieves the design momentum resolution for muons ejected towards the end-caps of ATLAS. Muons are important signatures of electroweak and Higgs sector events of interest for the ATLAS Collaboration's future physics goals~\cite{nsw_tdr}. Being the second of two tracking technologies on the NSWs, an effective alignment model of sTGC quadruplets is a necessary part of making the NSWs redundant for 10 or more years of recording collisions in the High Luminosity era of the LHC. 