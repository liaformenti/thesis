% ==================================================
% CHAPTER 8: Summary and Outlook %
% ==================================================

\chapter{Outlook and summary}
\label{chap:outlook_and_summary}
% Edit count: Lia - 1, Brigitte - 0


\subsubsection*{Outlook}
% --------------------------------------------------

The results presented in this thesis pave the way to the further application of the rich cosmic muon data set to alignment work.  First, a systematic study of cosmic ray and x-ray relative local strip position offsets should be performed for all quadruplets built for the NSWs. The correlation plots such as those presented in figure~\ref{fig:correlation} and~\ref{fig:no_correlation} can reveal unexpected results which could indicate an issue with either cosmic ray or x-ray data collection to be investigated further. Then, the overall correlation between x-ray and cosmic datasets should be quantified for all quadruplets instead of being quantified for each quadruplet individually.
 
For now, the correlation of the individual quadruplets tested supports the use of the x-ray data to build an alignment model. Work on the alignment model is ongoing~\cite{lefebvre_precision_2020}. Currently, the algorithm compares the offsets of a local group of strips at each x-ray gun position as measured by the x-ray and CMM methods in a fit to extract a global slope ($m$) and offset ($b$) per layer, $i$, where the $\chi^2$ is given by equation~\ref{eqn:chi2}.

\begin{equation}
    \chi^2 = \frac{\left[dy_{cmm, corr} - dy_{xray}\right]^2}{\delta dy_{xray}^2 + \delta dy_{cmm, corr}^2}\:,
    \label{eqn:chi2}  
\end{equation}
\begin{equation}
    dy_{cmm, corr} = y_{cmm} + b_i + m_{i}x - y_{nom}
    \label{eqn:dy_cmm_corr}
\end{equation}

Here, $dy$ is a local strip position offset as calculated from the x-ray and corrected CMM data and $\delta dy$ refers to their respective uncertainties. The plan is that the alignment parameters will be provided to the ATLAS experiment's offline software to reconstruct muon tracks from the NSWs' sTGCs. The large uncertainty on the x-ray local offsets (\SI{120}{\micro\meter}) and the sparseness of the measurements means that including input from other characterization datasets could reduce the uncertainty on the alignment model parameters. 

Relative local strip position offsets measured using cosmic ray data provide alignment information between the x-ray measurement points and can be calculated with a precision relevant to alignment studies. Therefore provide additional and complementary information that could further constrain the global rotation and translation parameter of the simple misalignement model currently being used. Work on using the three datatsets at once in a fit has begun.

%The CMM measurements were taken before the cathode boards were assembled into quadruplets, so alignment parameters for the given layer were extracted from the $\chi^2$ fit by stepping the corrected CMM $y$-position towards the x-ray $y$-position by adjusting the layer's slope and offset parameters.
%The uncertainty in the mean cosmics residuals was smaller than the desired position resolution of the sTGCs, so they provide relevant information about strip positions. Moreover, they can be calculated over the entire area of the quadruplet instead of at specific positions. It would be great to use the cosmics residuals as input to calculate and reduce the uncertainty on the alignment parameters. Since mean cosmics residuals can only provide relative alignment information, one idea would be to use them to constrain the fit of the alignment parameters. In this case, the alignment parameters would need to be fitted on all layers at once, and the shifting y-positions on each layer forced to create an abstracted track residual equal to the local mean cosmics residual (within uncertainty) for each x-ray point. Or, instead of constraining the fit, it could be penalized if the resulting parameters do not result in abstracted track residuals equal to the mean cosmics residuals within uncertainty. Some work on using the three datasets at once in a fit has been started.

\subsubsection*{Summary}
% --------------------------------------------------

The LHC~\cite{evans_lhc_2008} will be at the energy frontier of particle physics for at least the next decade, making it a unique tool with which to study particle physics. With the HL-LHC~\cite{hl_lhc_tdr}, high statistics on rare particle physics processes will enable more precise measurements of parameters of the Standard Model and increase the sensitivity to signatures of physics beyond the Standard Model~\cite{dainese_physics_2018}. To capitalize on the increased luminosity, the muon small wheels of the ATLAS experiment must be replaced to keep the triggering and tracking performance~\cite{nsw_tdr}. 

Small-strip thin gap chambers are gas ionization chambers optimized for a high rate environment~\cite{nsw_tdr}. Using the pad electrodes to define a region of interest makes it possible to get track segments of $\sim$\SI{1}{mrad} angular resolution quickly, which will be used as input to check if a collision originated from the interaction point and should be triggered on or not~\cite{nsw_tdr, perez-codina_small-strip_2016}. Thanks to the careful design of the gas gaps and the small pitch of the strip readout electrodes, sTGCs are also able to provide better than \SI{100}{\micro\meter} position resolution per detector plane to fulfill precision offline tracking requirements~\cite{abusleme_performance_2016}. 

Ultimately, the positions of the sTGC strip electrodes need to be known in ATLAS to within $\sim$\SI{100}{\micro\meter} so that they can deliver the required position resolution. The strategy is to build an alignment model to estimate the position of each strip. Input to the alignment model comes from the datasets used to characterize the quadruplets. The x-ray data~\cite{lefebvre_precision_2020} is the only characterization dataset that directly links the position of the strips to the ATLAS coordinate system. The alignment model could be built on x-ray data alone, but the sparseness of and large uncertainty on the local offsets mean that the alignment model could benefit from more input. The x-ray method is also a new technique that should be independently validated.

Relative local offsets measured with the cosmics and x-ray datasets were compared and the observed correlation confirmed the local offsets measured with the x-ray gun, validating the x-ray method. Moreover, the cosmics relative local offsets are useful on their own. The 2D visualizations of relative local offsets make it possible to quickly identify areas of misaligned strips and make hypotheses as to the physical origin of those misalignments. Also, the cosmics residual means were shown to be robust and have uncertainties under \SI{100}{\micro\meter} for all two-fixed-layer reference frames, which is small. Therefore, the cosmics relative local offsets complement the x-ray data by providing a complete, robust picture of the relative strip position offsets between layers. The next goal will be to use the cosmics relative local offsets to improve the alignment model and better position the sTGC strips in ATLAS.

Muons are important signatures of electroweak and Higgs sector events that physicists anticipate studying with a high-statistics dataset~\cite{dainese_physics_2018, nsw_tdr}. An effective alignment model of sTGC strip positions will ensure that the NSWs can be used to accomplish the ATLAS collaboration's physics goals during the High Luminosity era of the LHC.

% --------------------------------------------------
% \section{Importance in context}
% --------------------------------------------------
% \label{sec:importance}

%Ultimately, the positions of the sTGC strip electrodes need to be known in ATLAS to within $\sim$\SI{100}{\micro\meter} so that they can provide the required position resolution for the High-Luminosity LHC. The x-ray measurements will account for offsets of strips from their nominal position to achieve this goal. The cosmics dataset was used to confirm the local offsets measured with the x-ray gun, and could be used to improve the alignment model that will position each strip.

%Achieving the required position resolution on each layer of the NSWs in the particle track bending plane achieves the design momentum resolution for muons ejected towards the end-caps of ATLAS. Muons are important signatures of electroweak and Higgs sector events of interest for the ATLAS Collaboration's future physics goals~\cite{nsw_tdr}. Being the second of two tracking technologies on the NSWs, an effective alignment model of sTGC quadruplets is a necessary part of making the NSWs redundant for 10 or more years of recording collisions in the High Luminosity era of the LHC. 