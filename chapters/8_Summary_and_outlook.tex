% ==================================================
% CHAPTER 8: Summary and Outlook %
% ==================================================

\chapter{Outlook and summary}
\label{chap:outlook_and_summary}
% Edit count: Lia - 1, Brigitte - 0

The cosmic muon dataset was used to independently confirm the local offsets measured by the x-ray method. The x-ray offsets are being used to complete the sTGC alignment scheme of the NSWs: the NSW alignment system monitors the position of alignment platforms on the surface of sTGC wedges, and the x-ray measurements provide the offsets of the strip pattern with respect to each alignment platform. The continuation of this analysis is detailed next (section~\ref{sec:outlook}) before summarizing and considering the larger context (section~\ref{sec:summary}). 

% --------------------------------------------------
\section{Outlook}
% --------------------------------------------------
\label{sec:outlook}

Next all quadruplets with suitable cosmics and x-ray data should be surveyed to flag anomalous quadruplets (as a first step). If a quadruplet's correlation plot like figure~\ref{fig:correlation} or \ref{fig:no_correlation} reveals an unexpected correlation or has a large scatter, it would indicate an issue with either the cosmics or x-ray data collection to be investigated further. The uncertainty in each set of tracking points would inform the interpretation of the anomaly. Then, the quality of the correlation should be evaluated over all quadruplets instead of individually. 
 
For now, the correlation of the individual quadruplets tested support the use of the x-ray data to build an alignment model~\cite{lefebvre_precision_2020}. Work on creating an alignment model is ongoing. Currently, the algorithm compares the offsets of a local group of strips at each x-ray gun position as measured by the x-ray and CMM methods in a fit to extract a global slope ($m$) and offset ($b$) per layer, $i$, where the $\chi^2$ is given by equation~\ref{eqn:chi2}.

\begin{equation}
    \chi^2 = \frac{\left[dy_{cmm, corr} - dy_{xray}\right]^2}{\delta dy_{xray}^2 + \delta dy_{cmm, corr}^2}\:,
    \label{eqn:chi2}  
\end{equation}
\begin{equation}
    dy_{cmm, corr} = y_{cmm} + b_i + m_{i}x - y_{nom}
    \label{eqn:dy_cmm_corr}
\end{equation}

Here, $dy$ is an offset as calculated from the x-ray and corrected CMM data and $\delta dy$ refers to their respective uncertainties. The CMM measurements were taken before the cathode boards were assembled into quadruplets, so alignment parameters for the given layer were extracted from the $\chi^2$ fit by stepping the corrected CMM $y$-position towards the x-ray $y$-position by adjusting the layer's slope and offset parameters. The plan is that the alignment parameters will be provided to the ATLAS experiment's offline software to reconstruct muon tracks from the NSWs' sTGCs. The large uncertainty on the x-ray local offsets (\SI{120}{\micro\meter}) and the sparseness of the measurements means that including input from other characterization datasets could reduce the uncertainty on the alignment model parameters. 

The uncertainty in the mean cosmics residuals was smaller than the desired position resolution of the sTGCs, so they provide relevant information about strip positions. Moreover, they can be calculated over the entire area of the quadruplet instead of at specific positions. It would be great to use the cosmics residuals as input to calculate and reduce the uncertainty on the alignment parameters. Since mean cosmics residuals can only provide relative alignment information, one idea would be to use them to constrain the fit of the alignment parameters. In this case, the alignment parameters would need to be fitted on all layers at once, and the shifting y-positions on each layer forced to create an abstracted track residual equal to the local mean cosmics residual (within uncertainty) for each x-ray point. Or, instead of constraining the fit, it could be penalized if the resulting parameters do not result in abstracted track residuals equal to the mean cosmics residuals within uncertainty. Some work on using the three datasets at once in a fit has been started.

% --------------------------------------------------
\section{Summary}
% --------------------------------------------------
\label{sec:summary}

The LHC~\cite{evans_lhc_2008} will be at the energy frontier of particle physics for at least the next decade, making it a unique tool with which to study particle physics. With the HL-LHC~\cite{hl_lhc_tdr}, high statistics on rare particle physics processes will enable more precise measurements of parameters of the Standard Model and increase the sensitivity to signatures of physics beyond the Standard Model~\cite{dainese_physics_2018}. To capitalize on the increased collision rate, the NSWs of the ATLAS experiment must be replaced to keep the triggering and tracking performance~\cite{nsw_tdr}. 

Small-strip thin gap chambers are gas ionization chambers optimized for a high rate environment~\cite{nsw_tdr}. Using the pad electrodes to define a region of interest makes it possible to get track segments of $\sim$\SI{1}{mrad} angular resolution quickly, which will be used as input to check if a collision originated from the interaction point and should be triggered on or not~\cite{nsw_tdr, perez-codina_small-strip_2016}. sTGCs are also able to provide better than \SI{100}{\micro\meter} position resolution on each detector plane to fulfill precision offline tracking requirements~\cite{abusleme_performance_2016}. 

Ultimately, the positions of the sTGC strip electrodes need to be known in ATLAS to within $\sim$\SI{100}{\micro\meter} so that they can deliver the required position resolution. The ATLAS alignment system will position alignment platforms on the surface of the sTGC wedge, and an alignment model will be used to position the strips with respect to the alignment platforms~\cite{nsw_tdr}. Input to the alignment model comes from the datasets used to characterize the quadruplets. The x-ray method~\cite{lefebvre_precision_2020} is used to measure offsets of strips from their nominal position to achieve this goal. The alignment model could be built on x-ray data alone, but the sparseness of and large uncertainty on the local offsets mean that the alignment model could benefit from more input. Comparing the x-ray offsets to the CMM data~\cite{carlson_results_2019} allows the effect of inter-layer misalignments to be isolated and increases the input to the alignment model. 

The cosmics dataset was used to confirm the local offsets measured with the x-ray gun. It provides relative local offsets between sTGC strip layers. The 2D visualizations of relative local offsets allow personnel to quickly identify areas of misaligned strips and make hypotheses of the physical origin of those misalignments. The correlation seen between the x-ray and cosmics residuals in quadruplets with large relative misalignments confirms the validity of the x-ray local offsets. Moreover, the mean of track residuals in an area can be used to make a robust estimation of the relative local offset, as shown by the estimation of systematic uncertainties; the relative local offsets for all two-fixed layer reference frames do not change by more than \SI{100}{\micro\meter} given variation in data collection conditions and analysis algorithms. The cosmics relative local offsets are therefore relevant input for alignment studies and could improve the alignment model that will position each strip. 

Achieving the required position resolution on each layer of the NSWs in the particle track bending plane achieves the design momentum resolution for muons ejected towards the end-caps of ATLAS. Muons are important signatures of electroweak and Higgs sector events of interest for the ATLAS Collaboration's future physics goals~\cite{nsw_tdr}. Being the second of two tracking technologies on the NSWs, an effective alignment model of sTGC quadruplets layers is a necessary part of making the NSWs redundant for 10 or more years of recording collisions in the High Luminosity era of the LHC. 

% --------------------------------------------------
% \section{Importance in context}
% --------------------------------------------------
% \label{sec:importance}

%Ultimately, the positions of the sTGC strip electrodes need to be known in ATLAS to within $\sim$\SI{100}{\micro\meter} so that they can provide the required position resolution for the High-Luminosity LHC. The x-ray measurements will account for offsets of strips from their nominal position to achieve this goal. The cosmics dataset was used to confirm the local offsets measured with the x-ray gun, and could be used to improve the alignment model that will position each strip.

%Achieving the required position resolution on each layer of the NSWs in the particle track bending plane achieves the design momentum resolution for muons ejected towards the end-caps of ATLAS. Muons are important signatures of electroweak and Higgs sector events of interest for the ATLAS Collaboration's future physics goals~\cite{nsw_tdr}. Being the second of two tracking technologies on the NSWs, an effective alignment model of sTGC quadruplets is a necessary part of making the NSWs redundant for 10 or more years of recording collisions in the High Luminosity era of the LHC. 