% ==================================================
% CHAPTER 8: Outlook %
% ==================================================

\chapter{Outlook}
\label{chap:outlook}
% Edit count: Lia - 1, Brigitte - 0

The cosmic muon dataset can be used to independently confirm the absolute local offsets measured by the x-ray method. They are being used to complete the sTGC alignment scheme of the NSWs: the NSW alignment system monitors the position of alignment platforms on the surface of sTGC wedges, and the x-ray measurements provide the offsets of the strip pattern with respect to each alignment platform. The continuation of this analysis is detailed next (section~\ref{sec:next}) before considering the larger context (section~\ref{sec:importance}). 

% --------------------------------------------------
\section{Next steps}
% --------------------------------------------------
\label{sec:next}

Next all quadruplets with suitable cosmics and x-ray data should be surveyed to flag anomalous quadruplets (as a first step). If a quadruplet's correlation plot like figure~\ref{fig:correlation} or \ref{fig:no_correlation} reveals an unexpected correlation or has a large scatter, it would indicate an issue with either the cosmics or x-ray data collection to be investigated further. The uncertainty in each set of tracking points would inform the interpretation of the anomaly. Then, the quality of the correlation should be evaluated over all quadruplets instead of individually. 
 
For now, the correlation for the individual quadruplets tested support the use of the x-ray data to build a global misalignment model~\cite{lefebvre_precision_2020}. Work on creating a misalignment model is ongoing with the development of \package{stgc\_as\_built\_fit}~\cite{lefebvre_stgc_as_built_fit}. Currently, the algorithm compares the y-position of a local group of strips at each x-ray gun position as measured by the x-ray and CMM methods in a fit to extract a global slope ($m$) and offset ($b$) per layer, $l$, where the $\chi^2$ is given by equation~\ref{eqn:chi2}.

\begin{equation}
    dy_{cmm, corr} = y_{cmm} + b_l + m_{l}x - y_{nom}
    \label{eqn:dy_cmm_corr}
\end{equation}
\begin{equation}
    \chi^2 = \frac{\left[dy_{cmm, corr} - dy_{xray}\right]^2}{\delta dy_{xray}^2 + \delta dy_{cmm, corr}^2}
    \label{eqn:chi2}  
\end{equation}

Here, $dy$ refers to the corrected CMM and x-ray local offsets, and $\delta dy$ refers to their respective uncertainties. The CMM measurements were taken before the cathode boards were assembled into quadruplets, so alignment parameters for the given layer were extracted from the $\chi^2$ fit by stepping the corrected CMM y-position towards the x-ray y-position by adjusting the alignment parameters. The plan is that the alignment parameters will be provided to \package{Athena}~\cite{the_atlas_collaboration_athena} to precisely reconstruct muon tracks from the NSWs' sTGCs.

Figure~\ref{fig:correlation} and \ref{fig:no_correlation} clearly show that the uncertainty in the mean cosmics residuals was much smaller than the uncertainty in the x-ray residuals. It was the result of the uncertainty in the cosmic muon hit positions on each layer being smaller than the uncertainty on the x-ray beam profile centers (\SI{60}{\micro\meter} versus \SI{120}{\micro\meter}). Therefore, it would be great to use the cosmics residuals as input to calculate and reduce the uncertainty on the alignment parameters. Since mean cosmics residuals can only provide relative misalignment information, one idea would be to use them to constrain the fit of the alignment parameters. In this case, the alignment parameters would need to be fitted on all layers at once, and the shifting y-positions on each layer forced to create an abstracted track residual equal to the local mean cosmics residual (within uncertainty) for each x-ray point. Or, instead of constraining the fit, it could be penalized if the resulting parameters do not result in abstracted track residuals equal to the mean cosmics residuals within uncertainty. Some work on using the three datasets at once in a fit has begun but has prompted the idea to compare just the cosmics and CMM data first.

% --------------------------------------------------
\section{Importance in context}
% --------------------------------------------------
\label{sec:importance}

Ultimately, the positions of the sTGC strip electrodes need to be known in ATLAS to within ~\SI{100}{\micro\meter} so that they can provide the required position resolution for the High-Luminosity LHC. The x-ray measurements will account for misalignments between strip layers to achieve this goal. The cosmics dataset was used to confirm the local offsets measured with the x-ray gun, and could be used to improve the misalignment model that will position each strip.

Achieving the required position resolution on each layer of the NSWs in the particle track bending plane achieves the design momentum resolution for muons ejected towards the end-caps of ATLAS. Muons are important signatures of electroweak and Higgs sector events of interest for the ATLAS Collaboration's future physics goals~\cite{nsw_tdr}. Being the second of two tracking technologies on the NSWs, an effective misalignment model of sTGC quadruplets is a necessary part of making the NSWs redundant for 10 or more years of recording collisions in the High Luminosity era of the LHC. 