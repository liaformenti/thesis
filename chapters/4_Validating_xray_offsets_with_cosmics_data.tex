% ==================================================
% CHAPTER 4: Validating x-ray alignment parameters with cosmic muon data %
% ==================================================

\chapter{Validating x-ray alignment parameters with cosmic muon data}
\label{chap:comparison}
% Edit count: Lia - 0, Brigitte - 0

The goal of this work was to validate the alignment parameters extracted from the x-ray data with cosmics data. The complication was that the x-ray dataset provides absolute local offsets while the cosmics dataset provides relative local offsets, which cannot be compare directly. The solution was to analyze the x-ray data in the same relative coordinate system as the cosmics data.

% --------------------------------------------------
\section{Method for comparing x-ray and cosmics data}
% --------------------------------------------------
%TODO : determine if the description of equation 1.1 is suitably general enough to describe the x-ray data

The measured x-ray beam profile centers provided were systematically affected by local offsets in the same way as the mean cosmics residuals, as described by equation \ref{eqn:local_translation}. Therefore, if a 2-layer track is abstracted from the beam profile centers on each layer, and the residual calculated on a third layer, that residual should match the mean cosmics residual in that area. The track is "abstracted" because the beam profile center is actually the Gaussian mean of all selected cluster centroids recorded during the x-ray data taking period. This was the best analysis method because since the x-rays cause signal in the chamber via the photoeffect so there were not individual "x-ray tracks" to record. In fact the x-ray data was collected separately for each layer. However, since the effect of local offsets on the beam profile centers was the same, the difference in algorithm between x-ray and cosmics analysis did not matter. 

Therefore, for each x-ray survey position, the x-ray residual was calculated for all possible tracking combinations (which required an x-ray beam profile on at least three layers). The position of the x-ray residuals are shown as black dots over figure~\ref{fig:res_mean_th2_L2_F13} and \ref{fig:res_mean_th2_L4_F13}. Note that the mean of cosmics residuals around the x-ray points were calculated in bins exactly centered on the nominal x-ray gun position, unlike in figure~\ref{fig:res_mean_th2}.

%---------------------------------------------------
\section{Assessing correlation}
%---------------------------------------------------
Scatter plots of the x-ray and mean cosmics residuals for two sample quadruplets in figures~\ref{fig:correlation} and \ref{fig:no_correlation} reveal the degree of correlation between the datasets.

\begin{figure}
    \centering
    \includegraphics[width = \textwidth]{figures/figure_QL2P11_3100V_2021-08-05_QL2P11_local_cosmic_and_xray_data_correlation_plot.pdf}
    \caption{Correlation plot between x-ray and cosmics residuals for all tracking combinations for QL2.P.11. Each rectangle is centered on an x-ray and mean cosmics residual pair. The width of the rectangles in $x$ and $y$ are the uncertainty in the x-ray and mean cosmics residual respectively.}
    \label{fig:correlation}
\end{figure}

First, the fitted slope and offset in figure~\ref{fig:correlation} show that the two datasets largely supported one another for QL2.P.11. However, the magnitude of the uncertainties in the x-ray residuals is large, up to half a millimeter, since it comes from polating the measured x-ray beam centers which have uncertainty of \SI{120}{\micro\meter}. The large uncertainty sets a limit on the sensitivity of the analysis, for if the x-ray residuals of a quadruplet are smaller than the x-ray residual uncertainties, no conclusion about the correlation can be drawn, like for QL2.P.8 (figure~\ref{fig:no_correlation}).

\begin{figure}
    \centering
    \includegraphics[width = \textwidth]{figures/figure_QL2P08_3100V_2021-08-16_QL2P08_local_cosmic_and_xray_data_correlation_plot.pdf}
    \caption{Correlation plot between x-ray and cosmics residuals for all tracking combinations for QL2.P.8.}
    \label{fig:no_correlation}
\end{figure}

Several quadruplets were tested for each geometries: QL2P, QL2C, and QS3P. Each fell into one of the two categories: residuals large enough to see a correlation, or residuals too small too small to see a correlation. Since the x-ray and mean cosmics residuals are measures of the local relative offset between the layer of a quadruplet and the two reference layers, often the most relatively misaligned quadruplets have the largest range of residuals. So, the correlation plots are an easy visual way to identify quadruplets with large relative misalignments.

There are two patterns in the residuals on the scatter plot explained by geometry. First, residuals calculated through extrapolation tend to be larger because the extrapolation lever arm can produce more extreme values. Second, the pattern of points in figure~\ref{fig:correlation} is slightly mirrored. This is expected since the residuals calculated for a given set of three layers are geometrically correlated. Seeing these effects increases confidence in the analysis. 

The correlation of the cosmics residuals with the x-ray residuals alone does not validate the analysis method~---~all the checks in described in appendices~\ref{appendix:statistics} and ~\ref{appendix:systematics} do. The analysis could be validated externally by comparing the mean cosmics residuals to the relative misalignment parameters calculated using \package{tgc\_analysis/MatrixMethod} and JOHN FLORES CHI2 METHOD.

% --------------------------------------------------
\section{Limitations and next steps}
% --------------------------------------------------
The most important limit on measuring the degree of correlation between the x-ray and cosmics residuals is the uncertainty on the x-ray residuals, which stems from the uncertainty in the x-ray beam profile centers. The uncertainty mostly comes from systematic effects~\cite{lefebvre_precision_2020}, but the working group decided that the gains in confidence that more work on the method would provided were no longer a valuable use of time. They were limited primarily by the fact that sTGCs do not have good x-ray position resolution since x-rays do not create real tracks. 

The analysis for certain quadruplets was also limited by the availability of data. Sometimes, less than three layers were surveyed for a given x-ray gun position so no residuals could be calculated. Too few x-ray residuals prevent the analysis from detecting a significant correlation. Often, the analysis of smaller quadruplets suffered as a result because they have fewer possible gun positions in the first place. In addition, the analysis was limited to certain quadruplets because: the earliest constructed wedges (typically small wedges) were surveyed when the method was still under design and so have limited x-ray residuals calculated from beam profiles of lower quality; and data suitable for \package{tgc\_analysis/CosmicsAnalysis} was not collected for all quadruplets globally because \package{tgc\_analysis/CosmicsAnalysis} requires front end electronics for at least three gas volumes, which not all cosmic muon testing sites had.

Nonetheless, the comparison of x-ray and cosmics residuals was really to improve confidence in the x-ray data since it is used to define an absolute misalignment model for each quadruplet~\cite{lefebvre_precision_2020} to be input into \package{Athena}~\cite{the_atlas_collaboration_athena}~---~which this analysis succeeds in providing. Next all quadruplets with cosmics data suitable for \package{tgc\_analysis/CosmicsAnalysis} should be surveyed to weed out anomalous x-ray data and quantify the correlation between x-ray and cosmics data over all quadruplets instead of individually. For now, the results for individual quadruplets support the use of the x-ray data to build a global misalignment model. Work on creating a misalignment model is ongoing with the development of \package{stgc\_as\_built\_fit}~\cite{lefebvre_stgc_as_built_fit}. Currently, the algorithm compares the y-position of a local group of strips at each x-ray gun position as measured by the x-ray and CMM methods in a $\chi^2$ fit. The CMM measurements were taken before the cathode boards were assembled into quadruplets, so misalignment parameters for the given layer were extracted from the $\chi^2$ fit by stepping the CMM y-value towards the x-ray y-value by adjusting the misalignment parameters (currently, a rotation and global offset}. 

Figure~\ref{fig:correlation} and \ref{fig:no_correlation} clearly show that the uncertainty in the mean cosmics residuals, which is the sum in quadrature of the statistical and assigned systematic uncertainties (table~\ref{tab:sys_uncerts}), was much smaller than the uncertainty in the x-ray residuals. It is a result of the uncertainty in the cosmic muon hit positions on each layer being smaller than the uncertainty on the x-ray beam profile centers. Therefore, it would be great to use the cosmics residuals as input to calculate and reduce the uncertainty on the misalignment parameters. Since mean cosmics residuals can only provide relative misalignment information, one idea would be to constrain the fit of the alignment parameters by fitting the alignment parameters for all layers at once, and forcing the shifting y-values on each layer to result in abstracted track residuals equal to the mean cosmics residuals within their uncertainty. Or, instead of constraining the fit, it could be penalized if the resulting parameters do not result in abstracted track residuals equal to the mean cosmics residuals within uncertainty. 

