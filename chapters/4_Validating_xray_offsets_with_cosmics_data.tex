% ==================================================
% CHAPTER 4: Validating x-ray alignment parameters with cosmic muon data %
% ==================================================

\chapter{Validating x-ray alignment parameters with cosmic muon data}
\label{chap:comparison}
% Edit count: Lia - 0, Brigitte - 0

% --------------------------------------------------
\section{Presentation of theoretical method for comparison}
% --------------------------------------------------

The goal of this work is to validate the alignment parameters extracted from the x-ray data with cosmics data. The crux of the issue is that the x-ray dataset provides absolute local offsets while the cosmics dataset provides relative local offsets. The only solution is to analyze the x-ray data in the same relative coordinate system as cosmics.

The measured absolute beam profile center positions provided by the x-ray data are affected by local offsets the same way as cosmics (equation \ref{eqn:local_translation}). Therefore, if a 2-layer track is abstracted from the beam profile center positions on each layer, and the residual calculated on a third layer, that residual should match the cosmics residual. For example, the position of the x-ray residuals calculated for the x-ray track on layer X calculated from the beam profile centers on layers Y and Z for QL2.P.11 is shown \textit{in the last chapter or in the figure below! Yet to decide. Ask Brigitte}. 

%\begin{figure}
%    \centering
%    \includegraphics[width = \textwidth]{figures/potato.png}
%    \caption{Mean of residuals in each \SI{100}{\milli\meter} by \SI{100}{\milli\meter} bin over the area of the quad layer for QL2.P.11, with the position and value of the x-ray residuals plotted overtop.}
%    \label{fig:xray_beam_profile}
%\end{figure}

The track is abstracted because the beam profile center is actually the Gaussian mean of all selected cluster centroids that were recorded during the x-ray datataking, and in fact the x-ray data was collected seperately for each layer; but that does not matter for the purpose of this analysis. Several quadruplets of each Canadian geometry were tested using this method, and every possible tracking combination was used to get enough x-ray residuals to see the correlation. 

ROUGH

Note that the mean of cosmics residuals around the x-ray points were calculated in bins exactly centered on the nominal x-ray gun position. The area of the region of interest in which to include cosmics tracks to calculate the mean of cosmics residuals was chosen by . . . 

If the x-ray residuals were correlated with the cosmics residuals, it would validate the x-ray dataset.

% Comparing x-ray and cosmics residuals is precisely what was done with the new software package \package{strip_position_analysis}. Several quadruplets of each Canadian geometry were tested using this method, and every possible tracking combination was used to get enough comparison points to check the correlation. 

%TODO : include that x-ray data is collected layer by layer

% --------------------------------------------------
\section{Limitations}
% --------------------------------------------------
An x-ray gun position was only useful if data was recorded on at least 3 layers to be able to calculate a residual. As a result, the number of x-ray residuals for QL2s is around 70 while it is much smaller for QS3Ps.

