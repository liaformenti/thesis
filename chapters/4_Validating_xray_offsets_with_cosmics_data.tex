% ==================================================
% CHAPTER 4: Validating x-ray alignment parameters with cosmic muon data %
% ==================================================

\chapter{Validating x-ray alignment parameters with cosmic muon data}
\label{chap:comparison}
% Edit count: Lia - 0, Brigitte - 0

% --------------------------------------------------
\section{Presentation of theoretical method for comparison}
% --------------------------------------------------

The goal of this work is to validate the alignment parameters extracted from the x-ray data with cosmics data. 
% maybe delete this entire section if it is already elsewhere.

The x-ray parameters are in a useful coordinate system but the nature of x-ray events lead to large clusters and delta rays \cite{lefebvre_precision_2020}, and far less clusters are collected than in cosmics. Cosmic muon data is cleaner by the nature of the interaction of muons and less statistically limited, but the relative coordinate system cannot be used to calculate misalignment parameters. Therefore, it is desirable to validate the alignment parameters extracted from the x-ray dataset with cosmics data.

There is no reason the method of two-layer tracking cannot be applied to the x-ray data as well. The x-ray beam centroid on each layer can be taken as a track position, the track evaluated on another layer, then the residual calculated. As with cosmic muon clusters, the cluster position mean should be systematically offset by the local misalignment of the strip pattern, so when tracked the x-ray track residual should match the mean of cosmic track residuals. 

%TODO : include that x-ray data is collected layer by layer

