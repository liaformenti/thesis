% ==================================================
% Appendix: Analysis Systematics %
% ==================================================

%TODO : Organize figure positioning once you're done writing. 

\chapter[Analysis systematics]{Study of systematic uncertainties when using cosmics data for alignment studies}
\label{appendix:systematics}

% Sections:
% 2900 V vs 3100 V 
% Doub gaus vs gaus
% Area bin size
% Residual distribution bin size?
%TODO : Effect of clustering algorithm (reclustering) on residual means section.

% --------------------------------------------------
\section{Residual distribution fit function}
% --------------------------------------------------
\label{appendix:systematics_res_fit_fcn}

% Edit count: 1

The distribution of residuals should be modelled by a double gaussian fit\cite{lefebvre_thesis}:

\begin{equation}
\label{eqn:doub_gaus}
G(r) = A_{s}exp\left[ \frac{-(r-\mu)^{2}}{2\sigma_s^{2}} \right] + A_{b}exp\left[ \frac{-(r-\mu)^{2}}{2\sigma_b^{2}} \right]
\end{equation}

where $r$ is the residual, $A$ is the gaussian amplitude, $\mu$ is the gaussian mean, $\sigma$ is the gaussian sigma, and the subscripts $s$ and $b$ stand for signal and background respectively. One gaussian captures the real (signal) tracks and the other captures the tracks built from noise (background). The gaussian with the smaller width is identified as the signal. 

A single gaussian fit failed less often than a double gaussian fit. The gaussian fits were performed by initially estimating the amplitude to be 100 tracks, the gaussian mean to be the histogram mean, and gaussian $\sigma$ to be the RMS. The fit range was restricted to $\pm$1 RMS from the histogram mean. The modification helped the gaussian fit capture the signal peak. An example residual distribution is shown in figure~\ref{fig:double_gaussian_example_fit}. 

%TODO this doesn't need to be a full page figure.

\begin{figure}
    \centering
    \includegraphics[width = 0.85\textwidth]{figures/figure_double_gaus_vs_gaus_example_fit_QL2P08_3100V_2021-06-18_and_2021-07-19_xbin_10_ybin_5_layer4_fixedlayers12.pdf}
    \caption{Residual distribution for tracks on layer 4 built from hits on layers 1 and 2 for $x\in\left[-3.00, 97.00\right],  y\in\left[394.60, 494.60\right] mm$ for QL2.P.8 fit with a double gaussian and a single gaussian in a range of $\pm$1 RMS from the histogram mean.}
    \label{fig:double_gaussian_example_fit}
\end{figure}

For all residual distributions in \SI{100}{\milli\meter} by \SI{100}{\milli\meter} bins on layer 4 built from hits on layers 1 and 2, the difference in gaussian and double gaussian means and $\sigma$'s is shown in figure~\ref{fig:double_gaussian_compare_fits}. Since the RMS of the residual mean differences distribution is less than \SI{50}{\micro\meter} the gaussian fit gave the same result within the required precision. Moreover, this is for the tracking combination with the worst extrapolation lever arm and the widest distribution of mean differences; the interpolation combinations have narrower distributions. 

The gaussian $\sigma$ should be larger than the double gaussian $\sigma$ because the gaussian distribution includes the effect of the noise tracks with large residuals, while the double gaussian models signal and background residuals separately. For this analysis, only the residual mean was important, so the systematic overestimate of the signal $\sigma$ in the gaussian fit shown on the right of figure~\ref{fig:double_gaussian_compare_fits} was allowed.

\begin{figure}
    \centering
    \includegraphics[width = 0.8\textwidth]{figures/figure_compare_residual_fits_QL2P08_3100V_2021-06-18_no_dnl_minus_QL2P08_3100V_2021-07-19_doub_gaus_log_scale_layer4_fixedlayers12.pdf} 
    \caption{Difference in residual distribution means and $\sigma$'s for a gaussian and double gaussian fit, for all residual distributions in \SI{100}{\milli\meter} by \SI{100}{\milli\meter} bins on layer 4 built from hits on layers 1 and 2 for QL2.P.8.}
    \label{fig:double_gaussian_compare_fits}
\end{figure}

% --------------------------------------------------
\section{Cosmic muon data collection voltage}
% --------------------------------------------------
\label{appendix:systematics_2900V_vs_3100V}

% Edit count: 1

Cosmic muon data was collected at 2.9~kV and 3.1~kV because although 2.9~kV is closer to the operating conditions the chambers will be subject to in ATLAS, the extra gain provided by operating at 3.1~kV increased the signal to noise ratio for pad signals. Also, the tracking efficiency was higher with data collected at 3.1~kV. The difference in gain affected the relative population of clusters of different sizes, which in turn affected the uncertainty in the strip hit positions on each layer, the uncertainty in the track positions and the residual distributions. The residual distributions for 3.1~kV data are narrower, as shown in figure~\ref{fig:res_dist_2900V_3100V_412}.

\begin{figure}
    \centering
    \includegraphics[width = 0.7\textwidth]{figures/figure_residual_distributions_blue_QL2P08_2900V_2021-05-21_green_QL2P08_3100V_2021-05-21_layer4_fixedlayers12.pdf}
    \caption{Residual distribution for tracks on layer 4 built from hits on layers 1 and 2 for QL2.P.8 for data collected at 2.9~kV and 3.1~kV.}
    \label{fig:res_dist_2900V_3100V_412}
\end{figure}

Neither dataset is better for calculating the mean of residuals in a given area, so a systematic uncertainty can be assigned based on the difference in residual means calculated for 2.9~kV and 3.1~kV data; namely, the systematic uncertainty was approximated as the RMS of the residual mean difference distribution. Data taken with QL2.P.8 was used to estimate the RMS, as in figure~\ref{fig:voltage_compare_fits_412}.

Tracks built from hits on layers 1 and 2 and extrapolated to layer 4 have the worst lever arm and hence the most uncertainty. The width of the distribution for geometrically favourable tracks are much narrower. The narrowest width of the residual mean difference distribution is for tracks on layer 2 built from hits on layers 1 and 3 (see figure~\ref{fig:voltage_compare_fits_213}). 

Therefore, for each tracking combination, a systematic uncertainty equal to the RMS of the residual mean difference distribution was assigned.

\begin{figure}
\centering
\begin{subfigure}{.5\textwidth}
  \centering
  \includegraphics[width=\linewidth]{figures/figure_compare_residual_fits_QL2P08_2900V_2021-05-21_minus_QL2P08_3100V_2021-05-21_layer4_fixedlayers12_mean_differences.pdf}
  \caption{Tracks on layer 4, reference layers 1 and 2.}
  \label{fig:voltage_compare_fits_412}
\end{subfigure}%
\begin{subfigure}{.5\textwidth}
  \centering
  \includegraphics[width=\linewidth]{figures/figure_compare_residual_fits_QL2P08_2900V_2021-05-21_minus_QL2P08_3100V_2021-05-21_layer2_fixedlayers13_mean_differences.pdf}
  \caption{Tracks on layer 2, reference layers 1 and 3.}
  \label{fig:voltage_compare_fits_213}
\end{subfigure}
\caption{Difference in residual means for data collected with QL2.P.8 at 2.9~kV and 3.1~kV respectively in \SI{100}{\milli\meter} by \SI{100}{\milli\meter} bins for (a) tracks on layer 4 built from hits on layers 1 and 2 and (b) tracks on layer 2 built from hits on layers 1 and 3.}
\label{fig:voltage_compare_fits}
\end{figure}

% Effect of reclustering on residual means
% --------------------------------------------------
\section{Cluster fit algorithm}
% --------------------------------------------------
\label{appendix:systematics_cluster_fit_fcn}
To ensure that changing the cluster fitting algorithm like in appendix~\ref{appendix:clustering} would not change the calculated mean of residuals in each region of interest significantly, the residual means were compared in both cases. The distribution of the difference in residual means is plotted in figure~\ref{fig:cluster_fit_res_mean_compare_fits} for the tracking combination with the worst extrapolation lever arm.

\begin{figure}
    \centering
    \includegraphics[width = 0.7\textwidth]{figures/compare_residual_fits_QL2P08_3100V_2021-06-18_no_dnl_minus_QL2P08_3100V_2021-07-21_no_reclustering_layer4_fixedlayers12.pdf}
    \caption{Difference in residual means when the cluster fit algorithm is \package{Minuit2}~\cite{hatlo_developments_2005} versus Guo's method~\cite{guo_simple_2011} for tracks on layer 4 built from hits on layers 1 and 2 for QL2.P.8.}
    \label{fig:cluster_fit_res_mean_compare_fits}
\end{figure}

The other tracking combinations had smaller RMS values. Differences on the order of \SI{50}{\micro\meter} are important, so figure~\ref{fig:cluster_fit_res_mean_compare_fits} shows that the clustering algorithm had a small but notable effect. Therefore, the RMS for each tracking combination will be used to add a systematic uncertainty on the residual means.

% --------------------------------------------------
\section{Differential non-linearity}
% --------------------------------------------------
\label{appendix:systematics_dnl}
% Edit count: 1
In this context, differential non-linearity (DNL) is when the reconstructed cluster mean is biased by the fit of the discretely sampled PDO distribution over the strips. The bias depends on the relative position of the avalanche with respect to the center of the closest strip. For a summary of DNL, refer to page 40 of Lefebvre's thesis \cite{lefebvre_thesis}. The cluster mean was corrected for DNL using the equation:

\begin{equation}
\label{eqn:dnl_corr}
y' = y + a \sin \left( 2 \pi y_{rel} \right)
\end{equation}

where $y$ is the cluster mean, $y_{rel}$ is the relative position of the cluster mean with respect to the strip's center, $a$ is the amplitude of the correction, and $y'$ is the corrected cluster mean. The amplitude can be derived by comparing the reconstructed hit position to the expected hit position, as done in Abusleme, 2016 \cite{abusleme_performance_2016}. With cosmic muons, there is no reference hit position to compare to, so track residuals were used as a proxy \cite{lefebvre_thesis}. The hallmark of the DNL effect is the periodic pattern in the residual versus $y_{rel}$ profile, and the effect of correcting the cluster means using an amplitude of \SI{50}{\micro\meter} is shown in figure~\ref{fig:dnl_corr_effect}. An amplitude of \SI{50}{\micro\meter} was based on Lefebvre's estimate of the DNL amplitudes by layer, quadruplet and cluster size using exclusive cosmic muon tracks in \package{tgc\_analysis/CosmicsAnalysis}. Little variation was seen in the amplitude parameters with respect to the quadruplet tested, the layer and the cluster size so a universal correction was used.
%TODO How little is little? Check your notes.
%TODO exclusive better be defined somewhere clearly.

\begin{figure}
    \centering
    \includegraphics[width = 0.6\textwidth]{figures/figure_dnl_profiles_blue_QL2P08_3100V_2021-06-18_no_dnl_green_QL2P08_3100V_2021-06-18_2_50um_universal_DNL_layer4_fixed12.pdf}
    \caption{Effect applying a \SI{50}{\micro\meter} DNL correction to the cluster means on the residual vs $y_{rel}$ distribution for tracks built from layers 1 and 2 and extrapolated to layer 4 for QL2.P.8.}
    \label{fig:dnl_corr_effect}
\end{figure} 

Although the correction is not large enough in this case, the figure shows that the correction does reduce the DNL effect. Slightly better performance is seen in the interpolation tracking combinations where the quality of the residuals is better. DNL corrections for cosmic muon data are difficult because the DNL effect is obscured by the effect of misalignments and noise. Misalignments cause the center of the sine pattern in figure~\ref{fig:dnl_corr_effect} to be shifted off of zero, since the mean of residuals is shifted.

In figure~\ref{fig:dnl_compare_fits}, it is apparent that the effect of the DNL correction on the mean of the residual distribution in \SI{100}{\milli\meter} by \SI{100}{\milli\meter} areas is on the order of micrometers in the worst extrapolation case. Although the $\sigma$'s of the residual distributions shrink with the DNL correction, the mean is the parameter of interest. Therefore, for this analysis DNL was not corrected for.

\begin{figure}
    \centering
    \includegraphics[width = \textwidth]{figures/figure_compare_residual_fits_QL2P08_3100V_2021-06-18_no_dnl_minus_QL2P08_3100V_2021-06-18_2_50um_universal_DNL_layer4_fixedlayers12.pdf}
    \caption{Difference in residual distribution means and $\sigma$'s with and without DNL correction for residuals on layer 4 from reference layers 1 and 2 for QL2.P.8.}
    \label{fig:dnl_compare_fits}
\end{figure}

The $\sigma$'s of the residual distributions do shrink with the DNL correction but not so much to affect the residual means, which are the important parameter for this analysis. Therefore, since the effect of the DNL correction is negligible, it was not pursued further.






























