% T I T L E   P A G E
% -------------------
% Last updated October 23, 2020, by Stephen Carr, IST-Client Services
% The title page is counted as page `i' but we need to suppress the
% page number. Also, we don't want any headers or footers.
\pagestyle{empty}
\pagenumbering{roman}

% The contents of the title page are specified in the "titlepage"
% environment.
\begin{titlepage}
        \begin{center}
        \vspace*{1.0cm}

        \Huge
        {\bf For positioning electrodes in the ATLAS experiment's New Small Wheels using detector characterization techniques} \\

        \vspace*{1.0cm}

        \Large
        Lia Formenti \\
        
        \vspace*{1.0cm}
        
        \normalsize
        Department of Physics \\
        McGill University, Montreal \\
        October, 2021 \\

        \vspace*{3.0cm}

        \normalsize
        A thesis submitted to\\
        McGill University \\ 
        in partial fulfillment of the \\
        requirements of the degree of \\
        Master of Science \\

        \vspace*{2.0cm}

        \copyright\ Lia Formenti 2021 \\
        \end{center}
\end{titlepage}

% The rest of the front pages should contain no headers and be numbered using Roman numerals starting with `ii'
\pagestyle{plain}
\setcounter{page}{2}

\cleardoublepage % Ends the current page and causes all figures and tables that have so far appeared in the input to be printed.
% In a two-sided printing style, it also makes the next page a right-hand (odd-numbered) page, producing a blank page if necessary.

% T A B L E   O F   C O N T E N T S
% ---------------------------------
\renewcommand\contentsname{Table of Contents}
\tableofcontents
\cleardoublepage
\phantomsection    % allows hyperref to link to the correct page

% A B S T R A C T
% ---------------

\begin{center}\textbf{Abstract}\end{center}
% Medium length abstract
% The particle collision rate at the Large Hadron Collider (LHC) will be effectively increased in 2025-2027 by an extensive upgrade program. The upgrades will improve the statistics on measurements and the sensitivity of searches for rare processes using the ATLAS experiment and other experiments at the LHC. The innermost endcaps of the ATLAS muon spectrometer consist of two wheels of muon detectors that must be replaced to maintain the muon momentum resolution in the high-rate environment. The so-called New Small Wheels (NSWs) are covered with two detector technologies: micromegas and small-strip thin gap chambers (sTGCs). Canada is responsible for 1/4 of the required sTGCs. sTGCs are gas ionization chambers that hold a thin volume of gas between two cathode boards. One board is segmented into strips of \SI{3.2}{mm} pitch that are used to precisely measure the coordinate of a passing muon. Four sTGCs glued together, called a quadruplet, cover the NSWs. Quadruplets were designed to achieve \SI{1}{mrad} angular resolution to fulfill the spectrometer's precision tracking and triggering requirements. A requirement to deliver the angular resolution is positioning the strips in the ATLAS coordinate system to within the chambers' position resolution (less than \SI{100}{\micro\meter}). The ATLAS alignment system is able to position the surface of the quadruplets, so the interal geometry of the quadruplets must be characterized. At McGill University, quadruplets are characterized using a cosmic ray hodoscope before being sent to CERN, where the charge profile left by x-rays is used to measure the local offset of the strip pattern at specific positions on the quadruplet surface. The x-ray method has acceptable but limited precision. It is being used to position the strips within the ATLAS alignment system. Given the importance of alignment, the x-ray method must be validated by an external method. Cosmic ray data is used to characterize the relative alignment between layers and validate the x-ray method.

% Too long version of abstract (rough)
% The collision rate in the LHC will be effectively increased in 2025-2027 by an extensive upgrade program. The innermost endcaps of the ATLAS muon spectrometer consist of two wheels of muon detectors that must be replaced to improve the angular resolution of tracks for precision muon momentum reconstruction. The New Small Wheels (NSWs) will be covered with two detector types that must trigger on and track outgoing particles: micromegas and small-strip thin gap chambers (sTGCs). Canada is responsible for one quarter of the required sTGCs.  sTGCs are gas ionization chambers that hold a thin volume of gas between two cathode boards. One board is segmented into strips of \SI{3.2}{mm} pitch that are used to measure the precision coordinate. At McGill University, modules with four layers of sTGCs called quadruplets are characterized using a cosmic ray hodoscope before being sent to CERN for further testing and integration into the wheels. Quadruplets must be able to reconstruct particle tracks with 1 mrad angular resolution using the precision coordinate recorded by the strips of each sTGC layer for ATLAS' physics goals. A requisite to delivering the angular resolution is positioning the strips in the ATLAS coordinate system to within the chambers' position resolution (\SI{100}{\micro\meter}). The ATLAS alignment system will be able to position alignment platforms on the surface of quadruplets, so the internal geometry of the chambers must be measured and corrected for. Analyzing the residuals of cosmic ray tracks is used to measure the offset of the strip pattern on one layer with respect to other layers in areas of interest. Looking at the relative offsets over the surface of an sTGC layer characterizes the quadruplets' relative alignment. To get the strip pattern offsets in ATLAS' absolute coordinate system, the charge profile left by an x-ray gun is used to measure the offset of the strip pattern from nominal. These offsets are used to create an alignment model for each sTGC strip layer. The x-ray measurements are limited to the positions of the alignment platform and have limited precision, so their accuracy must be verified by an independent method. In this work, cosmic ray data is used to study the relative alignment between quadruplet strip layers and to validate the x-ray method. 

% and coordinate measuring machine (CMM) measurements of strip cathode boards are being used to define alignment parameters.
The particle collision rate at the Large Hadron Collider (LHC) will be increased up to seven times its design value in 2025-2027 by an extensive upgrade program. The innermost endcaps of the ATLAS muon spectrometer consist of two wheels of muon detectors that must be replaced to maintain the muon momentum resolution in the high-rate environment. The so-called New Small Wheels (NSWs) are covered with two detector technologies: micromegas and small-strip thin gap chambers (sTGCs). sTGCs are gas ionization chambers that hold a thin volume of gas between two cathode boards. One board is segmented into strips of \SI{3.2}{mm} pitch that are used to precisely reconstruct the coordinate of a passing muon. Modules of four sTGCs glued together into quadruplets cover the NSWs. Quadruplets were designed to achieve \SI{1}{mrad} angular resolution to fulfill the spectrometer's triggering and precision tracking requirements. To deliver the angular resolution the strips must be positioned in the ATLAS coordinate system to within the chambers' position resolution (less than \SI{100}{\micro\meter}). So, the internal geometry of the quadruplets must be characterized. At McGill University, quadruplets were characterized using a cosmic ray hodoscope before being sent to CERN, where the charge profile left by x-rays is used to measure the offset of the strip patterns at known positions on the quadruplet surface. The x-ray measurements are being used to position the strips within the ATLAS alignment system. They have acceptable but limited precision and do not span the whole area of the strip layers. Given the importance of alignment, the x-ray method must be validated by an independent method. Cosmic ray data is used to characterize the relative alignment between layers and validate the x-ray method.

\cleardoublepage

% R E S U M E
% ---------------

\begin{center}\textbf{R\'{e}sum\'{e}}\end{center}

Le rythme des collisions du collisioneur LHC serra augmonter jusqu'\'{a} sept fois le rythme de cr\'{e}ation en 2025-2027 par un programme d'am\'{e}lioration important. Une partie du spectrom\`{e}tre \'{a} muons est compos\'{e} de deux roues couvertes de d\'{e}tecteurs de muons qui doivent \^{e}tre remplacer pour mantenir la r\'{e}solution d\'{e}lan apr\`{e}s l'augmentation du rythme. Les nouvelles petites roues (NSWs) comme elles sont appel\'{e}es sont couvertes de deux technologies: des d\'{e}tecteurs micromegas et des chambres sTGC (chambres \'{a} petites bandes et \'{a} intervalles fins). Les sTGCs sont des chambres d'ionisation de gaz, qui contient un volume fin entre deux panneux cathodiques. Un panneau est segment\'{e} \'{a} petites bandes avec un pente de \SI{3.2}{mm} qui sont utilis\'{e}es pour reconstruire pr\'{e}cisement le coordonn\'{e}e d'un muon qui passe. Des modules de quatres sTGCs coll\'{e}es ensembles en quaduplets couvrent les NSWs. Les quadruplets \'{e}taient con\c{c}us pour r\'{e}aliser une r\'{e}solution angulaire de \SI{1}{mrad} pour satisfaires les besoins des syst\`{e}mes de d\'{e}clenchement et de mesures de pr\'{e}cision. Pour r\'{e}aliser la r\'{e}solution angulaire les bandes doivent \^{e}tre position\'{e}es dans ATLAS avec une pr\'{e}cision moin de \SI{100}{\micro\meter}, la r\'{e}solution de position des sTGCs. Alors, la g\'{e}ometrie internelles des quadruplets doivent \^{e}tre charactaris\'{e}s. \'{A} l'Universit\'{e} de McGill, les quadruplets \'{e}taient tester avec des rayons cosmiques avant \^{e}tre envoye \`{a} CERN, o\`{u} le profile de charge fait par des rayons-X est utilis\'{e} pour m\'{e}surer le d\'{e}placement du motif des bandes \`{a} des positions specifiques sur le surface des quadruplets. Les d\'{e}placements m\'{e}surer pas les rayons-X sont utilis\'{e}s pour positioner les bandes dedans le syst\`{e}me d'alignement d'ATLAS. Ils peuvent \^{e}tre m\'{e}surer avec un pr\'{e}cision acceptable mais limit\'{e}, et ne couvrent pas la r\'{e}gion enti\`{e}re du panneau. \'{E}tant donn\'{e} l'importance de l'alignement, la m\'{e}thode des rayons-X doit \^{e}tre valider par une m\'{e}thode ind\'{e}pendente. Le donn\'{e}es ramass\'{e}es avec les rayons cosmiques sont utilis\'{e}es pour charactariser l'alignement relative entre les panneaux et pour valider la m\'{e}thode des rayons-X.

% A C K N O W L E D G E M E N T S
% -------------------------------

\begin{center}\textbf{Acknowledgements}\end{center}

Something along the lines of . . . I would like to thank all the little people who made this thesis possible.
\cleardoublepage

% C O N T R I B U T I O N
% -------------------------------
  % The following is a sample Delaration Page as provided by the GSO
  % December 13th, 2006.  It is designed for an electronic thesis.
 \begin{center}\textbf{Contribution of authors}\end{center}
  
 \noindent
Something along the lines of . . . I hereby declare that I am the sole author of this thesis. This is a true copy of the thesis, including any required final revisions, as accepted by my examiners.

\cleardoublepage

% L I S T   O F   F I G U R E S
% -----------------------------
% \addcontentsline{toc}{chapter}{List of Figures}
% \listoffigures
% \cleardoublepage
% \phantomsection		% allows hyperref to link to the correct page

% L I S T   O F   T A B L E S
% ---------------------------
% \addcontentsline{toc}{chapter}{List of Tables}
% \listoftables
% \cleardoublepage
% \phantomsection		% allows hyperref to link to the correct page

% Change page numbering back to Arabic numerals
\pagenumbering{arabic}

