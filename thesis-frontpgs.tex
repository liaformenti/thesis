% T I T L E   P A G E
% -------------------
% Last updated October 23, 2020, by Stephen Carr, IST-Client Services
% The title page is counted as page `i' but we need to suppress the
% page number. Also, we don't want any headers or footers.
\pagestyle{empty}
\pagenumbering{roman}

% The contents of the title page are specified in the "titlepage"
% environment.
\begin{titlepage}
        \begin{center}
        \vspace*{1.0cm}

        \Huge
        {\bf Cosmic ray validation of electrode positions in small-strip thin gap chambers for the upgrade of the ATLAS detector } \\

        \vspace*{1.0cm}

        \Large
        Lia Formenti \\
        
        \vspace*{1.0cm}
        
        \normalsize
        Department of Physics \\
        McGill University, Montreal \\
        October, 2021 \\

        \vspace*{3.0cm}

        \normalsize
        A thesis submitted to\\
        McGill University \\ 
        in partial fulfillment of the \\
        requirements of the degree of \\
        Master of Science \\

        \vspace*{2.0cm}

        \copyright\ Lia Formenti 2021 \\
        \end{center}
\end{titlepage}

% The rest of the front pages should contain no headers and be numbered using Roman numerals starting with `ii'
\pagestyle{plain}
\setcounter{page}{2}

\cleardoublepage % Ends the current page and causes all figures and tables that have so far appeared in the input to be printed.
% In a two-sided printing style, it also makes the next page a right-hand (odd-numbered) page, producing a blank page if necessary.

% T A B L E   O F   C O N T E N T S
% ---------------------------------
\renewcommand\contentsname{Table of Contents}
\tableofcontents
\cleardoublepage
\phantomsection    % allows hyperref to link to the correct page

% A B S T R A C T
% ---------------

\begin{center}\textbf{Abstract}\end{center}
% Medium length abstract
% The particle collision rate at the Large Hadron Collider (LHC) will be effectively increased in 2025-2027 by an extensive upgrade program. The upgrades will improve the statistics on measurements and the sensitivity of searches for rare processes using the ATLAS experiment and other experiments at the LHC. The innermost endcaps of the ATLAS muon spectrometer consist of two wheels of muon detectors that must be replaced to maintain the muon momentum resolution in the high-rate environment. The so-called New Small Wheels (NSWs) are covered with two detector technologies: micromegas and small-strip thin gap chambers (sTGCs). Canada is responsible for 1/4 of the required sTGCs. sTGCs are gas ionization chambers that hold a thin volume of gas between two cathode boards. One board is segmented into strips of \SI{3.2}{mm} pitch that are used to precisely measure the coordinate of a passing muon. Four sTGCs glued together, called a quadruplet, cover the NSWs. Quadruplets were designed to achieve \SI{1}{mrad} angular resolution to fulfill the spectrometer's precision tracking and triggering requirements. A requirement to deliver the angular resolution is positioning the strips in the ATLAS coordinate system to within the chambers' position resolution (less than \SI{100}{\micro\meter}). The ATLAS alignment system is able to position the surface of the quadruplets, so the interal geometry of the quadruplets must be characterized. At McGill University, quadruplets are characterized using a cosmic ray hodoscope before being sent to CERN, where the charge profile left by x-rays is used to measure the local offset of the strip pattern at specific positions on the quadruplet surface. The x-ray method has acceptable but limited precision. It is being used to position the strips within the ATLAS alignment system. Given the importance of alignment, the x-ray method must be validated by an external method. Cosmic ray data is used to characterize the relative alignment between layers and validate the x-ray method.

% Too long version of abstract (rough)
% The collision rate in the LHC will be effectively increased in 2025-2027 by an extensive upgrade program. The innermost endcaps of the ATLAS muon spectrometer consist of two wheels of muon detectors that must be replaced to improve the angular resolution of tracks for precision muon momentum reconstruction. The New Small Wheels (NSWs) will be covered with two detector types that must trigger on and track outgoing particles: micromegas and small-strip thin gap chambers (sTGCs). Canada is responsible for one quarter of the required sTGCs.  sTGCs are gas ionization chambers that hold a thin volume of gas between two cathode boards. One board is segmented into strips of \SI{3.2}{mm} pitch that are used to measure the precision coordinate. At McGill University, modules with four layers of sTGCs called quadruplets are characterized using a cosmic ray hodoscope before being sent to CERN for further testing and integration into the wheels. Quadruplets must be able to reconstruct particle tracks with 1 mrad angular resolution using the precision coordinate recorded by the strips of each sTGC layer for ATLAS' physics goals. A requisite to delivering the angular resolution is positioning the strips in the ATLAS coordinate system to within the chambers' position resolution (\SI{100}{\micro\meter}). The ATLAS alignment system will be able to position alignment platforms on the surface of quadruplets, so the internal geometry of the chambers must be measured and corrected for. Analyzing the residuals of cosmic ray tracks is used to measure the offset of the strip pattern on one layer with respect to other layers in areas of interest. Looking at the relative offsets over the surface of an sTGC layer characterizes the quadruplets' relative alignment. To get the strip pattern offsets in ATLAS' absolute coordinate system, the charge profile left by an x-ray gun is used to measure the offset of the strip pattern from nominal. These offsets are used to create an alignment model for each sTGC strip layer. The x-ray measurements are limited to the positions of the alignment platform and have limited precision, so their accuracy must be verified by an independent method. In this work, cosmic ray data is used to study the relative alignment between quadruplet strip layers and to validate the x-ray method. 

% and coordinate measuring machine (CMM) measurements of strip cathode boards are being used to define alignment parameters.
The Large Hadron Collider (LHC) is used to generate subatomic physics processes at the energy frontier to challenge our understanding of the Standard Model of particle physics. The particle collision rate at the LHC will be increased up to seven times its design value in 2025-2027 by an extensive upgrade program. The innermost endcaps of the ATLAS muon spectrometer consist of two wheels of muon detectors that must be replaced to maintain the muon momentum resolution in the high-rate environment. The so-called New Small Wheels (NSWs) are made of two detector technologies: micromegas and small-strip thin gap chambers (sTGCs). The sTGCs are gas ionization chambers that hold a thin volume of gas between two cathode boards. One board is segmented into copper readout strips of \SI{3.2}{mm} pitch that are used to precisely reconstruct the coordinate of a passing muon. Modules of four sTGCs glued together into quadruplets cover the NSWs. Quadruplets were designed to achieve a \SI{1}{mrad} angular resolution to fulfill the spectrometer's triggering and precision tracking requirements. To achieve the required angular resolution the absolute position of the readout strips must be known in the ATLAS coordinate system to within \SI{100}{\micro\meter}. At McGill University, the performance of sTGC quadruplets was characterized using cosmic ray data before being sent to CERN, where the charge profile left by x-rays is used to measure the offset of the strip patterns with respect to nominal at a limited number of points on the surface of each quadruplet. The x-ray strip position measurements have acceptable but limited precision and do not span the whole area of the strip layers. Given the importance of knowing the absolute position of each readout strip to achieve the performance requirements of the NSWs, the x-ray method must be validated by an independent method. Cosmic ray data is used to characterize the relative alignment between layers and validate the x-ray method.

\cleardoublepage

% R E S U M E
% ---------------

\begin{center}\textbf{R\'{e}sum\'{e}}\end{center}

Le grand collisioneur des hadrons (LHC) utilise des collisions de protons afin de g\'{e}n\'{e}rer des processus de la physique subatomique \`{a} la fronti\`{e}re m\^{e}me de la haute \'{e}nergie, et ceci afin de tenter remettre en cause le mod\`{e}le standard de la physique des particules. Le taux des collisions entre protons au LHC sera augmont\'{e} jusqu'\`{a} sept fois le taux nominal d'ici 2025-2027 \`{a} l'aide d'un programme de mise \`{a} niveau de grande envergure. Une partie du spectrom\`{e}tre \`{a} muons du d\'{e}tecteur ATLAS consistant de deux roues de d\'{e}tecteurs de muons doit \^{e}tre remplac\'{e}e afin de mantenir la r\'{e}solution sur l'inertie des muons \`{a} haut taux de collision. Appel\'{e}es les Nouvelles Petites Roues (NSWs), elles utilisent deux technologies de d\'{e}tection differentes: des chambres micromegas et des chambres \`{a} petites bandes et \`{a} intervalles fins (sTGCs). Les sTGCs sont des chambres d'ionisation de gaz, qui contiennent un volume tr\`{e}s fin de gaz entre deux panneux cathodiques. Un panneau est segment\'{e} avec de petites bandes en cuivre en pente de \SI{3.2}{mm}. Ceux-ci d\'{e}tectent le signal laiss\'{e} par des muons et permettent la mesure pr\'{e}cise des coordonn\'{e}es spatiales des muons qui traversent le d\'{e}tecteur. Des modules de quatre sTGCs coll\'{e}es ensemble en quaduplets couvrent la superficie des NSWs. Ces quadruplets ont \'{e}t\'{e} con\c{c}us afin de permettre une r\'{e}solution angulaire de \SI{1}{mrad}, et de satisfaire les exigences des syst\`{e}mes de d\'{e}clenchement et de mesures de pr\'{e}cision. Afin d'atteindre cette r\'{e}solution angulaire il faut que la position absolue de chaque bande soit connue au sein du d\'{e}tecteur ATLAS avec une pr\'{e}cision d'au moins \SI{100}{\micro\meter}. \`{A} l'Universit\'{e} de McGill, la performance des quadruplets a \'{e}t\'{e} caract\'{e}riser avec des rayons cosmiques avant leur envoi au CERN, o\`{u} le profil des charges laiss\'{e} par des rayons X est utilis\'{e} pour mesurer le d\'{e}placement du motif des bandes par rapport \`{a} leur emplacement nominal. Ceci est fait \`{a} un nombre de positions limit\'{e} sur la surface des quadruplets. Ces d\'{e}placements, mesur\'{e}s par les rayons X, ont une pr\'{e}cision acceptable mais limit\'{e}e et ne couvrent pas la r\'{e}gion enti\`{e}re des panneaux. \'{E}tant donn\'{e} l'importance de la caract\'{e}risation pr\'{e}cise de la position absolue de chaque bande afin de r\'{e}aliser les exigences de rendement des NSWs, une m\'{e}thode ind\'{e}pendente de validation de la m\'{e}thode des rayons X est requise. Les donn\'{e}es recuellies avec les rayons cosmiques sont utilis\'{e}es pour caract\'{e}riser l'alignement relatif entre les panneaux et valider la m\'{e}thode des rayons X.

\cleardoublepage

% A C K N O W L E D G E M E N T S
% -------------------------------

\begin{center}\textbf{Acknowledgements}\end{center}

Experimental particle physics projects are never done alone. I am grateful to have been working with the ATLAS Collaboration for two years now.

Thank you to Dr. Brigitte Vachon for her guidance throughout this project and for editing this thesis. I am consistently amazed by her ability to jump into the details of my project and discuss them with me. She has also supported me as a whole person, encouraging me to pursue volunteering for science outreach and consider opportunities I may not have found on my own.

Thanks also to Dr. Benoit Lefebvre, who collected some of the data used in this thesis, wrote several software tools I used to analyze the data and advised me several times throughout this project. 

Thank you to my labmates at McGill University, Dr. Tony Kwan, Kathrin Brunner, John McGowan and Charlie Chen. Kathrin taught me mechanical skills that I had not learned otherwise and that I will apply elsewhere. She also is my model of a thoughtful, careful and organized experimentalist. Tony, manager of the laboratory, created the most encouraging, trusting and productive work environment I have ever been a part of.

Thank you to the friends I can call on at anytime, and thank you to my family whose constant support makes every step possible.

\cleardoublepage

% C O N T R I B U T I O N
% -------------------------------
  % The following is a sample Delaration Page as provided by the GSO
  % December 13th, 2006.  It is designed for an electronic thesis.
 \begin{center}\textbf{Contribution of authors}\end{center}
  
 \noindent

I, the author, was involved in collecting the cosmic ray data from September 2019 - March 2021. I did not design the cosmic ray testing procedure nor write the data preparation software, but I participated in using the software to analyze cosmic ray results. In the thesis, the terms ``clustering,'' ``local offset'' and ``relative local offset'' are defined. Cosmic ray clustering was done in the data preparation software, but I redid the fit afterwards to explore sensitivity to the fit algorithm. With help from Dr. Lefebvre and Dr. Vachon, I helped design the software that calculated the relative local offsets from cosmic ray data. I wrote that software on my own. I was not involved in the design, data collection, data preparation or analysis of the x-ray data. I also was not involved in creating an alignment model from the x-ray data. I used the x-ray local offsets calculated using x-ray data analysis software to calculate relative local offsets with x-rays. I did the comparison between the x-ray and cosmic ray data.

I hereby declare that I am the sole author of this thesis. This is a true copy of the thesis, including any required final revisions, as accepted by my examiners.

\cleardoublepage

% L I S T   O F   F I G U R E S
% -----------------------------
% \addcontentsline{toc}{chapter}{List of Figures}
% \listoffigures
% \cleardoublepage
% \phantomsection		% allows hyperref to link to the correct page

% L I S T   O F   T A B L E S
% ---------------------------
% \addcontentsline{toc}{chapter}{List of Tables}
% \listoftables
% \cleardoublepage
% \phantomsection		% allows hyperref to link to the correct page

% Change page numbering back to Arabic numerals
\pagenumbering{arabic}

