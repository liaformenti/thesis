% T I T L E   P A G E
% -------------------
% Last updated October 23, 2020, by Stephen Carr, IST-Client Services
% The title page is counted as page `i' but we need to suppress the
% page number. Also, we don't want any headers or footers.
\pagestyle{empty}
\pagenumbering{roman}

% The contents of the title page are specified in the "titlepage"
% environment.
\begin{titlepage}
        \begin{center}
        \vspace*{1.0cm}

        \Huge
        {\bf For positioning electrodes in the ATLAS experiment's New Small Wheels using detector characterization techniques} \\

        \vspace*{1.0cm}

        \Large
        Lia Formenti \\
        
        \vspace*{1.0cm}
        
        \normalsize
        Department of Physics \\
        McGill University, Montreal \\
        October, 2021 \\

        \vspace*{3.0cm}

        \normalsize
        A thesis submitted to\\
        McGill University \\ 
        in partial fulfillment of the \\
        requirements of the degree of \\
        Master of Science \\

        \vspace*{2.0cm}

        \copyright\ Lia Formenti 2021 \\
        \end{center}
\end{titlepage}

% The rest of the front pages should contain no headers and be numbered using Roman numerals starting with `ii'
\pagestyle{plain}
\setcounter{page}{2}

\cleardoublepage % Ends the current page and causes all figures and tables that have so far appeared in the input to be printed.
% In a two-sided printing style, it also makes the next page a right-hand (odd-numbered) page, producing a blank page if necessary.

% T A B L E   O F   C O N T E N T S
% ---------------------------------
\renewcommand\contentsname{Table of Contents}
\tableofcontents
\cleardoublepage
\phantomsection    % allows hyperref to link to the correct page

% A B S T R A C T
% ---------------

\begin{center}\textbf{Abstract}\end{center}

%TODO This is your CAP virtual congress / myThesis abstract. Rewrite for final.

ATLAS is a multi-purpose particle detector system designed to capture the outcome of proton-proton collisions at the Large Hadron Collider (LHC) at CERN. The innermost endcaps of the ATLAS muon spectrometer consist of two wheels of muon detectors that must be replaced to improve the angular resolution of tracks for precision muon momentum reconstruction in the next phase of LHC operation. The New Small Wheels (NSWs) will be covered with two detector types that must trigger on and track outgoing particles - one type is small-strip thin gap chambers (sTGCs). Canada is responsible for one quarter of the required sTGCs. At McGill University, modules with four layers of sTGCs (called quadruplets) are characterized using a cosmic ray hodoscope before being sent to CERN for further testing and integration into the wheels. Quadruplets must be able to reconstruct particle tracks with 1 mrad angular resolution. Misalignments between sTGC layers must be corrected for to achieve this goal. The charge profile left by an x-ray gun and coordinate measuring machine (CMM) measurements of quadruplet layers are being used to define these parameters. Work on using cosmic ray data to validate misalignment parameters derived using the above-mentioned methods will be presented.

\cleardoublepage

% R E S U M E
% ---------------

\begin{center}\textbf{R\'{e}sum\'{e}}\end{center}

C'est le r\'{e}sum\'{e}.

Ad eros odio amet et nisl in nostrud consequat iusto eum suscipit autem vero enim dolore exerci, ut. Esse ex, magna in facilisis duis amet feugait augue accumsan zzril. Lobortis aliquip dignissim at, in molestie nibh, vulputate feugait nibh luptatum ea delenit nostrud dolore minim veniam odio. Euismod delenit nulla accumsan eum vero ullamcorper eum ad velit veniam. Quis, exerci ea feugiat nulla molestie, veniam nonummy nulla. Elit tincidunt, consectetuer dolore nulla ipsum commodo, ut, at qui blandit suscipit accumsan feugiat vel praesent.

Dolor zzril wisi quis consequat in autem praesent dignissim, sit vel aliquam at te, vero. Duis molestie consequat eros tation facilisi diam dolor augue. Dolore dolor in facilisis et facilisi et adipiscing suscipit eu iusto praesent enim, euismod consectetuer feugait duis vulputate.

\cleardoublepage

% A C K N O W L E D G E M E N T S
% -------------------------------

\begin{center}\textbf{Acknowledgements}\end{center}

Something along the lines of . . . I would like to thank all the little people who made this thesis possible.
\cleardoublepage

% C O N T R I B U T I O N
% -------------------------------
  % The following is a sample Delaration Page as provided by the GSO
  % December 13th, 2006.  It is designed for an electronic thesis.
 \begin{center}\textbf{Contribution of authors}\end{center}
  
 \noindent
Something along the lines of . . . I hereby declare that I am the sole author of this thesis. This is a true copy of the thesis, including any required final revisions, as accepted by my examiners.

\cleardoublepage

% L I S T   O F   F I G U R E S
% -----------------------------
% \addcontentsline{toc}{chapter}{List of Figures}
% \listoffigures
% \cleardoublepage
% \phantomsection		% allows hyperref to link to the correct page

% L I S T   O F   T A B L E S
% ---------------------------
% \addcontentsline{toc}{chapter}{List of Tables}
% \listoftables
% \cleardoublepage
% \phantomsection		% allows hyperref to link to the correct page

% Change page numbering back to Arabic numerals
\pagenumbering{arabic}

