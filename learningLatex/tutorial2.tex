\documentclass[12]{article}
\begin{document}

Superscripts: 
$$2x^2$$
$$2x^{24}$$
$$2x^{3x+4}$$
$$2x^{3x^4+5}$$

Subscripts:
$$x_1$$
$$x_{12}$$
$${{x_1}_2}_3$$

Greek letters:
$$\pi$$
$$\alpha$$
$$\omega$$
$$\Omega$$

Trig functions:
$$y=\sin(x)$$ % Notice the slash
$$y=\log(x)$$
$$y=\ln(x)$$
$$y=\log_5(x)$$

Square roots:
$$\sqrt{2}$$
$$\sqrt[3]{2}$$
$$\sqrt{2x+1}$$
$$\sqrt{1 + \sqrt{x}}$$

Fractions:\\
% Math mode keeps it slanted.
About $2/3$rds of the glass is full.\\
% To get small horizontal inline fractions,
About $\frac{2}{3}$ of the glass is full.
% If we want the fraction inline but larger,
About $\displaystyle\frac{2}{3}$ of the glass is full.\\
% Use tabs to jump between fcn args
% Completing fcn with tabs gives you the right number of arguments.
$$\frac{x}{x^2+2x+1	}$$
$$\frac{x+1}{\sqrt[3]{x+3}}$$
$$\frac{1}{\frac{1}{x+1}}$$
$$\frac{1}{1+\frac{1}{x}}$$
$$\sqrt[•]{\frac{1}{x^2+1}}$$



\end{document}