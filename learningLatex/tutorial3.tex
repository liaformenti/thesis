\documentclass[12pt]{article}
\usepackage{amsfonts,amssymb,amsmath}
\usepackage{float}
\parindent 0px % Don't indent my paragraphs.
\pagestyle{empty} % turning off page numbers

\begin{document}
The distributive property states that $a(b+c)=ab+ac$ for all $a,b,c\in\mathbb{R}$.\\[6pt] %6pt for adding extra spacing.
The equivalence class of $a$ is $[a]$.\\
% Curly brackets won't show up in math mode.
% You have to backslash them.
The set $A$ is defined to be $\{1,2,3\}$.\\
The cost is $\$10.00$.\\
% To auto resize the brackets for math mode, use \left( and \right)
$$2\left(\frac{1}{x^2-1}\right)$$
$$2\left\{\frac{1}{x^2-1}\right\}$$
% Angular brackets
$$2\left\langle\frac{1}{x^2-1}\right\rangle$$
% Abs value (use pipe)
$$2\left|\frac{1}{x^2-1}\right|$$
% One-sided brackets -\left needs \right. Use period after \left to not draw left bracket.
$$\left.\frac{dy}{dx}\right|_{x=1}$$
$$\left(\frac{1}{1+\left(\frac{1}{1+x}\right)}\right)$$

Tables:
% c stands for centered, as in you want text centered in your column.
% Other options are l and r for left and right.

\begin{tabular}{|c||c|c|c|c|c|}
\hline
$x$ & 1 & 2 & 3 & 4 & 5 \\  % end of row
\hline % delineating horizontal line
$f(x)$ & 10 & 11 & 12 & 13 & 14 \\
\hline
\end{tabular}

\vspace{1cm}

% Note: the compiler chooses its "best" position for the table. 
% To get out of this, use [H]. Need float package.
\begin{table}[H]
\centering % centre the table
\def\arraystretch{3} % easy way to vertically pad table cells.
\begin{tabular}{|c||c|c|c|c|c|}
\hline
$x$ & 1 & 2 & 3 & 4 & 5 \\  % end of row
\hline % delineating horizontal line
$f(x)$ & $\frac{1}{2}$ & 11 & 12 & 13 & 14 \\
\hline
\end{tabular}
\caption{These values are a sample of the function $f(x)$.}
\end{table}

% Fancy table
% Tables get auto numbered, tabular alone don't.
\begin{table}[H]
\centering % centre the table
\def\arraystretch{1.5} % easy way to vertically pad table cells.
\begin{tabular}{|c|c|}
\hline
$f'(x)$ & $f(x)$ \\  % end of row
\hline % delineating horizontal line
$x>0$ & The function $f(x)$ is increasing. \\
\hline
\end{tabular}
\end{table}

% BLAH BLAH BLAH long sentences in tables.
\begin{table}[H]
\centering % centre the table
\caption{Table with a really long sentence.} % Notice the caption is on the top now.
\def\arraystretch{1.5} % easy way to vertically pad table cells.
\begin{tabular}{|l|p{2in}|} % This is where you make sure the long line gets wrapped.
\hline
$f'(x)$ & $f(x)$ \\  % end of row
\hline % delineating horizontal line
$x>0$ & The function $f(x)$ is increasing. What a beautiful thing! OMG I'm really enjoying this tutorial. Free thought is useful. Good day to you. \\
\hline
\end{tabular}
\end{table}

Arrays:
% Inside align, you're automatically in math mode.
% To get text using \text
% You can force spaces in math mode with \space.
\begin{align}
5x^2 \space\text{My words here}
\end{align}

Arrays:
% Inside align, you're automatically in math mode.
% To get text using \text
% You can force spaces in math mode with \space.
% To line up equals signs put use &=
% Throws an error if you backslash the last array line of align block.
% Auto numbers each eqn of the array (carries forward to next eqns).
% Use asterix to kill auto numbering.
\begin{align*}
5x^2-9&=x+3\\
5x^2-x-12&=0
\end{align*}

\end{document}